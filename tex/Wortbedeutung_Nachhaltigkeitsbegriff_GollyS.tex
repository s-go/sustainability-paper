\documentclass[
    german,
    a4paper,%
    %11pt,%
    12pt,%
    oneside,%
    %twoside,%
    %titlepage,%
    %liststotoc,%
    toc=bibliography,
    %bibtotoc,%
    %headinclude,%
    %draft,
    final,
    %pointlessnumbers,%
    %fleqn,% mathematische Gleichungen linksbündig statt zentriert
]{scrartcl}

\usepackage{polyglossia} % (neue) deutsche Beschriftungen und Silbentrennung
\setdefaultlanguage[spelling=new]{german}
%\usepackage{times} % Nimbus Roman statt CM Serif
\usepackage{lmodern} % Latin Modern (in T1) statt CM
%%\usepackage[T1]{fontenc} % T1-Kodierung: Umlaute als *eine* Glyphe
%%\usepackage[utf8]{inputenc}
\usepackage{fontspec}
\setsansfont{Myriad Pro}
\setmainfont{Adobe Garamond Pro}

%%%%%%%%%%%%%%%% Seitenspiegel, Typografie %%%%%%%%%%%%%%%%%%%
%\usepackage{pdflscape} % Querformat: \begin{landscape}
\usepackage{geometry} % Seitenränder selbst bestimmen
\geometry{a4paper,%
          top=18mm,%
          left=20mm,%
          right=20mm,% ohne Marginalien: 20mm - mit Marginalien: 45mm
          bottom=22mm,%
          headsep=10mm,%
          footskip=12mm,%
         }
\setlength{\parindent}{0pt} % kein Einruecken bei Absatzbeginn
\setlength{\parskip}{8pt} % Absaetze durch Abstand kennzeichnen (1 Zeichenhoehe)

% Space between footnote mark and text
\usepackage[hang]{footmisc}
\setlength{\footnotemargin}{1em}

%\setlength{\marginparwidth}{3cm} % Marginalien
%\usepackage{caption}
%
%\captionsetup{
%  format=plain,
%  margin=0em,
%  labelsep=newline,
%  justification=raggedright
%}

\usepackage{ulem} % durchgestrichener Text: \sout{}
\usepackage[usenames,dvipsnames,svgnames,table]{xcolor} % schönere Farben, z. B. RawSienna
\usepackage{enumerate} % Aufzählungsstil anpassen, z. B. {enumerate}[a)] – \setcounter{enumi}{4}

\usepackage{natbib} % \bibitem[Guevara(2010)]{Guevara2010} - \citet[5]{Guevara2010}
\bibpunct{(}{)}{;}{a}{}{,} % Interpunktion in Zitaten
\setcitestyle{notesep={: }} % Doppelpunkt zwischen Jahr und Seitenzahl

\usepackage{tocstyle}
\newtocstyle[KOMAlike][leaders]{alldotted}{}
\usetocstyle{alldotted}

%%%%%%%%%%%%%%%% Quelltext-Satz %%%%%%%%%%%%%%%%%%%
\usepackage{textcomp} % Text Companion fonts (für einfache Anführungszeichen)
\usepackage{listings} % Umgebung lstlisting; \lstinline$...$
\lstset{
	language=,                % the language of the code
	basicstyle=\ttfamily\small,
	xleftmargin=2em,
	xrightmargin=2em,
	captionpos=b,
	abovecaptionskip=.5em,
	commentstyle=\color{OliveGreen},% sets comment style
	tabsize=3,                      % sets default tabsize
	breaklines=true,                % sets automatic line breaking
	breakatwhitespace=true,         % sets if automatic breaks should only happen at whitespace
	showspaces=false,               % show spaces adding particular underscores
	showstringspaces=false,         % underline spaces within strings
	escapechar=§,                   % escapes to LaTeX
	columns=flexible,               % columns=fixed / columns=flexible / columns=fullflexible
	upquote=true,                   % straight quotes
	literate={ö}{{\"o}}1            % national characters:
			 {ä}{{\"a}}1            %   *{replace}{replacement text}{length in output}
			 {ü}{{\"u}}1            %   * (optional): not in delimited text (strings, comments, ...) 
			 {Ö}{{\"O}}1
			 {Ä}{{\"A}}1
			 {Ü}{{\"U}}1
			 {ß}{{\ss}}2
	}

%%%%%%%%%%%%%%%%%%% Linguistik-Pakete %%%%%%%%%%%%%%%%%%%%%%%%
\usepackage{mathtools} % includes amsmath, adds some nice fixes
\usepackage{amssymb} % für \varnothing
%\usepackage{semantic} % für |[ |]
%\usepackage{qtree} % qtree \Tree [. ] [. ] \qroof{}.
%\usepackage[x11names, rgb]{xcolor} % für dot2tex
%\usepackage{tikz} % für dot2tex
%\usetikzlibrary{arrows,shapes} % für dot2tex
         %%% unbedingt nach tikz-Paketen laden %%%
%\usepackage{gb4e} % Beispiel-Umgebung: \begin{exe} \ex \begin{xlist}
%\usepackage{avm}             % für AVMs
%\avmfont{\sc}                % allgemeine AVM-Schriftart
%\avmvalfont{\it}             % Schriftart für Werte
%\avmsortfont{\footnotesize\it} % Schriftart für Sort Labels 

%%%%%%%%%%%%%%%%%%% hyperref %%%%%%%%%%%%%%%%%%%%%%%%
\usepackage[hidelinks]{hyperref} % letzter Paketaufruf!
\makeatletter % changes the catcode of @ to 11
\AtBeginDocument{
  \hypersetup{ % hyperref: \title und \author in PDF-Eingenschaften übernehmen
    pdftitle = {\@title},
    pdfsubject = {\@subject},
    pdfauthor = {\@author}
  }
}
\makeatother % changes the catcode of @ back to 12

%%%%%%%%%%%%%%%%%%%%%%% Titel %%%%%%%%%%%%%%%%%%%%%%%%%%
\subject{Korpuslinguistische Untersuchungen zur Veränderung von Wortbedeutungen durch Medienge- und -missbrauch}
% TODO: Revise title
\title{Der Nachhaltigkeitsbegriff – multifuktionale Zauberformel oder inhaltsleeres Plastikwort?}
\subtitle{}
%\setkomafont{author}{\normalsize}
\setkomafont{date}{\normalsize}
\setkomafont{publishers}{\normalsize}
\author{Sebastian Golly\\ {\normalsize (761737)}}
\date{\today}

% Double line spacing for text body, but not for titles
% TODO: Activate double line spacing
\usepackage[onehalfspacing]{setspace}
\addtokomafont{disposition}{\linespread{1}}

% No extra line spacing between list items
\usepackage{enumitem}
\setlist{nosep}

% URL definitions
\usepackage{url}
\urldef{\urlcorpussize}\url{http://kaskade.dwds.de/dstar/public/hist.perl?fmt=hist&pformat=svg&q=%24l%3D%2Fnachhaltig%2Fi&_s=submit&n=date%2Bclass&smooth=none&sg=1&grid=1&sl=10%2B7&w=0&wb=0&pr=0&xr=1777%3A2016&yr=0%3A*&psize=840%2C480&T=1}
\urldef{\urlwortverlaufgesamt}\url{https://www.dwds.de/r/plot?view=3&norm=date%2Bclass&smooth=spline&genres=0&grand=1&slice=10&prune=0&window=0&wbase=0&logavg=0&logscale=0&xrange=1776%3A2016&q1=%24l%3D%2Fnachhaltig%2Fi}
% \urldef{\urlXXX}\url{}


\begin{document}
%%%%%%%%%%%%%%%%%%%%%%%%%%%%%%%%%%%%%%%%%%%%%%%%%%%%%%%%%%%%%%
\maketitle

\vfill

\paragraph{Abstract}

% TODO: Write abstract
TODO 
\\[3em]

\vfill

\begin{center}
Universität Potsdam\\[1.5em]
Sommersemester 2017
\end{center}

\thispagestyle{empty}
\newpage

%%%%%%%%%%%%%%%%%%%%%%%%%%%%%%%%%%%%%%%%%%%%%%%%%%%%%%%%%%%%%%

\section{Einleitung}
\label{sec:einleitung}

% TODO: Write paper


\section{Relevante Literatur}
\label{sec:rel-literatur}


\section{Der Nachhaltigkeitsbegriff – Entwicklung und Definition}
\label{sec:entwicklung-definition}

\subsection{Ursprung}
\label{subsec:ursprung}

\subsection{Entwicklung}
\label{subsec:entwicklung}

\subsection{Definition}
\label{subsec:definition}

\subsection{Kritik}
\label{subsec:kritik}

\section{Ziele der Studie}
\label{sec:ziele}

Wie wir in Abschnitt \ref{subsec:kritik} gesehen haben, kritisieren bereits seit den 90er Jahren verschiedene Wissenschaftler, dass die Benennung \textit{nachhaltig} so inflationär und unscharf verwendet wird, dass ihre fachsprachliche Bedeutung (etwa im Sinne der Brundtland-Definition) mehr und mehr verwässert und im Diskurs zugunsten einer blassen allgemeinsprachlichen Bedeutung (als Synonym zu \textit{dauerhaft}, \textit{längere Zeit anhaltend}) verloren geht. Allerdings gibt es meines Wissens bisher keine korpuslinguistische Untersuchung, die diese These empirisch überprüft. Die vorliegende Arbeit soll diese Lücke schließen.

Zu diesem Zweck soll sie folgende Fragen untersuchen:

\begin{enumerate}
\item Wie hat sich die Verwendungshäufigkeit der Benennung \textit{nachhaltig} seit ihrem ersten Auftreten entwickelt?
\item Wie haben sich in diesem Zeitraum die Kontexte verändert, in denen \textit{nachhaltig} auftritt?
\item Wird die Benennung \textit{nachhaltig} in verschiedenen Zeitabschnitten eher in ihrer fachsprachlichen Bedeutung (im Sinne der Brundtland-Definition), in ihrer allgemeinsprachlichen Bedeutung (als Synonym zu \textit{dauerhaft}, \textit{längere Zeit anhaltend}) oder gemischt verwendet?
\end{enumerate}

Korpuslinguistisch fundierte Antworten auf diese Fragen sollen dazu beitragen, die Diskussion zu versachlichen, und einen Impuls weg von einer wenig konstruktiven kulturpessimistisch-sprachkritischen Argumentation geben.

\section{Methodik}
\label{sec:methodik}

Die Beantwortung der Fragestellungen aus Abschnitt~\ref{sec:ziele} erfordert eine diachrone korpuslinguistische Untersuchung auf Grundlage eines Korpus, das möglichst die gesamte Zeitspanne vom ersten Auftreten der Benennung \textit{nachhaltig} bis heute abdeckt.

Es gibt nicht viele Korpora des Deutschen, die diese Anforderung erfüllen. Die Wahl fiel auf die Referenz- und Zeitungskorpora (aggregiert) des Digitalen Wörterbuchs der deutschen Sprache (DWDS). Diese aggregierte Textsammlung vereint die folgenden Korpora:

\begin{itemize}
\item Deutsches Textarchiv (1600–1900) \citep[vgl.][]{Geyken2011}
\item DWDS-Kernkorpus (1900–1999) \citep[vgl.][]{Geyken2007}
\item DWDS-Kernkorpus 21 (2000–2010) \citep[vgl.][]{Geyken2007}
\item die Zeitungskorpora des DWDS, bestehend aus
	\begin{itemize}
	\item Berliner Zeitung (1946–2005)
	\item neues deutschland (1946–1990)
	\item Der Tagesspiegel (1996–2005)
	\item Die ZEIT (1946–2016)
	\end{itemize}
\end{itemize}

Damit deckt das aggregierte Korpus die gesamte Zeitspanne von 1600 bis 2016 mit einer großen Bandbreite verschiedener Textsorten und Disziplinen ab. Da die älteste bekannte Verwendung der Benennung \textit{nachhaltig} auf das Jahr 1713 datiert \citep[vgl.][99]{Zürcher1965}, ist das Korpus für die Zwecke dieser Studie gut geeignet.

Um verschiede Aspekte der Fragestellungen zu untersuchen, kamen unterschiedliche Werkzeuge zum Einsatz:

\begin{itemize}
\item \textbf{DiaCollo}\footnote{Verfügbar unter \url{http://kaskade.dwds.de/dstar/public/diacollo/}.} \citep{Jurish2015} für Frequenz- und Kollokationsanalysen,
\item \textbf{DWDS-Wortverlaufskurve}\footnote{Verfügbar unter \url{https://www.dwds.de/r/plot}.} zur Visualisierung von Frequenzverteilungen,
\item \textbf{DWDS-Korpusabfrage}\footnote{Verfügbar unter \url{https://www.dwds.de/r}.} zur Kontextanalyse von Korpusbelegen.
\end{itemize}

Für alle Analysen wurde \lstinline|$l=/nachhaltig/i| als (Teil der) Anfrage verwendet. Sie matcht alle Tokens, in deren Lemma die Zeichenkette „nachhaltig“ vorkommt (unabhängig von Groß- und Kleinschreibung), also etwa „nachhaltig“, „unnachhaltig“, „nachhaltigste“, „Nachhaltigkeit“, „UN-Nachhaltigkeitsziel“ oder „Weltnachhaltigkeitsgipfel“.

Der früheste Korpusbeleg für diese Anfrage datiert auf das Jahr 1779. Um die Auswertungen in glatte Zehnjahreszeiträume segmentieren zu können, wurde der Gesamtzeitrahmen der Analyse auf den Zeitraum von 1776 bis 2016 festgelegt. Für die detaillierten Kontext- und Kollokationsanalysen wurde er in vier Epochen untergliedert:

\begin{itemize}
\item 1776–1845
\item 1846–1915
\item 1916–1985
\item 1986–2016
\end{itemize}

Zu beachten ist allerdings, dass die Textmenge im Korpus über die Zeit nicht konstant ist. Abbildung~\ref{fig:corpus-size} zeigt die Verteilung der Anzahl von Tokens im Korpus pro Zehnjahreszeitraum seit 1776. Beim diachronen Vergleich von Auftretenshäufigkeiten einer Benennung oder Kollokation sind absolute Frequenzen deshalb kein aussagekräftiger Indikator. Stattdessen wird in dieser Arbeit vor allem die normalisierte Frequenz pro Million Tokens als Metrik verwendet.

\begin{figure}[h!]
\centering
\includegraphics[width=.8\textwidth]{img/dstar-public-corpus-size-1777-2016-edit}
\caption[corpus-size]{Anzahl von Tokens im Korpus pro Zehnjahreszeitraum seit 1776\footnotemark}
\label{fig:corpus-size}
\end{figure}
\footnotetext{Erstellt mit D*/public Time Series: \urlcorpussize, abgerufen am 08.09.2017.}

\section{Ergebnisse}
\label{sec:ergebnisse}

\subsection{Frequenzanalyse im Gesamtzeitraum}
\label{subsec:freq-gesamt}

Insgesamt gibt es im Korpus 28.230 Belege für Tokens, in deren Lemma die Zeichenkette „nachhaltig“ vorkommt. Der früheste Korpusbeleg datiert auf das Jahr 1779, der späteste auf 2016.

Abbildung~\ref{fig:nachhaltig-freq-1776-2016} zeigt die normalisierten Auftretenshäufigkeiten von \textit{nachhaltig} und verwandten Wörtern im Zeitraum zwischen 1776 und 2016. Während sich die Frequenz bis 1945 im Bereich zwischen 0 und 10 Vorkommen pro Million Tokens bewegt, steigt sie bis 1986 leicht auf 17,28 Vorkommen pro Million Tokens. Ab 1996 kommt es zu einem starken Anstieg der Verwendungshäufigkeit, die im Zeitraum 2006–2015 den Spitzenwert von 41.04 erreicht. Im Jahr 2016 fällt die normalisierte Frequenz schließlich auf 33,48 Vorkommen pro Million Tokens ab.

\begin{figure}[h!]
\centering
\includegraphics[width=.8\textwidth]{img/nachhaltig-freq-1776-2016}
\caption[corpus-size]{DWDS-Wortverlaufskurve für \lstinline|$l=/nachhaltig/i| im Zeitraum 1776–2016\footnotemark}
\label{fig:nachhaltig-freq-1776-2016}
\end{figure}
\footnotetext{Erstellt durch das Digitale Wörterbuch der deutschen Sprache: \urlwortverlaufgesamt, abgerufen am 05.09.2017.}

\begin{table}[h!]
\centering
\renewcommand{\arraystretch}{1.5}

\caption{Roh- und normalisierte Frequenzen (pro Million Tokens) von \textit{nachhaltig} in den untersuchten Epochen}
\label{tab:freq-gesamt}

\begin{tabular}{ccc}
\textbf{Epoche} & $\mathbf{f_{roh}}$ & $\mathbf{f_{norm}}$ \\ \hline
1776–1845 & 206 & 3,70 \\ \hline
1846–1915 & 606 & 6,89 \\ \hline
1916–1985 & 3071 & 12,91 \\ \hline
1986–2016 & 24.356 & 29,63 \\ \hline
\end{tabular} 

\end{table}

Dieser Frequenzverlauf belegt einen deutlichen Anstieg der Verwendungshäufigkeit von \textit{nachhaltig} seit Mitte der 90er Jahre. Ob diese quantitative Entwicklung mit einer Bedeutungsverschiebung einhergeht, soll im Folgenden eine detaillierte Analyse der Kontexte und Kollokationen in den einzelnen Epochen zeigen.

\subsection{Detailanalyse 1776–1845}
\label{subsec:detail-1776–1845}



\section{Diskussion}
\label{sec:diskussion}


\section{Zusammenfassung}
\label{sec:zusammenfassung}



\newpage
\begin{thebibliography}{9}

% TODO: Add references

\bibitem[Jurish(2015)]{Jurish2015} Jurish, B. (2015). DiaCollo: On the trail of diachronic collocations. In \textit{Proceedings of the CLARIN Annual Conference}. 28–31.
\bibitem[Geyken(2007)]{Geyken2007} Geyken, A. (2007). The DWDS corpus: A reference corpus for the German language of the 20th century. In: Fellbaum, C. (Hg.): \textit{Collocations and Idioms: Linguistic, lexicographic, and computational aspects.} London, 23–41.
\bibitem[Geyken et al.(2011)]{Geyken2011} Geyken, A., Haaf, S., Jurish, B., Schulz, M., Steinmann, J., Thomas, C., \& Wiegand, F. (2011). Das Deutsche Textarchiv: Vom historischen Korpus zum aktiven Archiv. \textit{Digitale Wissenschaft}, 157.
\bibitem[Zürcher(1965)]{Zürcher1965} Zürcher, U. (1965). \textit{Die Idee der Nachhaltigkeit unter spezieller Berücksichtigung der Gesichtspunkte der Forsteinrichtung.} Dissertation, ETH Zürich.

\end{thebibliography}

\end{document}
