\documentclass[
    german,
    a4paper,%
    %11pt,%
    12pt,%
    oneside,%
    %twoside,%
    %titlepage,%
    %liststotoc,%
    toc=bibliography,
    %bibtotoc,%
    %headinclude,%
    %draft,
    final,
    %pointlessnumbers,%
    %fleqn,% mathematische Gleichungen linksbündig statt zentriert
]{scrartcl}

\usepackage{polyglossia} % (neue) deutsche Beschriftungen und Silbentrennung
\setdefaultlanguage[spelling=new]{german}
%\usepackage{times} % Nimbus Roman statt CM Serif
\usepackage{lmodern} % Latin Modern (in T1) statt CM
%%\usepackage[T1]{fontenc} % T1-Kodierung: Umlaute als *eine* Glyphe
%%\usepackage[utf8]{inputenc}
\usepackage{fontspec}
\setsansfont{Myriad Pro}
\setmainfont{Adobe Garamond Pro}

%%%%%%%%%%%%%%%% Seitenspiegel, Typografie %%%%%%%%%%%%%%%%%%%
%\usepackage{pdflscape} % Querformat: \begin{landscape}
\usepackage{geometry} % Seitenränder selbst bestimmen
\geometry{a4paper,%
          top=18mm,%
          left=20mm,%
          right=20mm,% ohne Marginalien: 20mm - mit Marginalien: 45mm
          bottom=22mm,%
          headsep=10mm,%
          footskip=12mm,%
         }
\setlength{\parindent}{0pt} % kein Einruecken bei Absatzbeginn
\setlength{\parskip}{8pt} % Absaetze durch Abstand kennzeichnen (1 Zeichenhoehe)

% Space between footnote mark and text
\usepackage[hang]{footmisc}
\setlength{\footnotemargin}{1em}

%\setlength{\marginparwidth}{3cm} % Marginalien
%\usepackage{caption}
%
%\captionsetup{
%  format=plain,
%  margin=0em,
%  labelsep=newline,
%  justification=raggedright
%}

\usepackage{ulem} % durchgestrichener Text: \sout{}
\usepackage[usenames,dvipsnames,svgnames,table]{xcolor} % schönere Farben, z. B. RawSienna
\usepackage{enumerate} % Aufzählungsstil anpassen, z. B. {enumerate}[a)] – \setcounter{enumi}{4}

\usepackage{natbib} % \bibitem[Guevara(2010)]{Guevara2010} - \citet[5]{Guevara2010}
\bibpunct{(}{)}{;}{a}{}{,} % Interpunktion in Zitaten
\setcitestyle{notesep={: }} % Doppelpunkt zwischen Jahr und Seitenzahl

\usepackage{tocstyle}
\newtocstyle[KOMAlike][leaders]{alldotted}{}
\usetocstyle{alldotted}

%%%%%%%%%%%%%%%% Quelltext-Satz %%%%%%%%%%%%%%%%%%%
\usepackage{textcomp} % Text Companion fonts (für einfache Anführungszeichen)
\usepackage{listings} % Umgebung lstlisting; \lstinline$...$
\lstset{
	language=,                % the language of the code
	basicstyle=\ttfamily\small,
	xleftmargin=2em,
	xrightmargin=2em,
	captionpos=b,
	abovecaptionskip=.5em,
	commentstyle=\color{OliveGreen},% sets comment style
	tabsize=3,                      % sets default tabsize
	breaklines=true,                % sets automatic line breaking
	breakatwhitespace=true,         % sets if automatic breaks should only happen at whitespace
	showspaces=false,               % show spaces adding particular underscores
	showstringspaces=false,         % underline spaces within strings
	escapechar=§,                   % escapes to LaTeX
	columns=flexible,               % columns=fixed / columns=flexible / columns=fullflexible
	upquote=true,                   % straight quotes
	literate={ö}{{\"o}}1            % national characters:
			 {ä}{{\"a}}1            %   *{replace}{replacement text}{length in output}
			 {ü}{{\"u}}1            %   * (optional): not in delimited text (strings, comments, ...) 
			 {Ö}{{\"O}}1
			 {Ä}{{\"A}}1
			 {Ü}{{\"U}}1
			 {ß}{{\ss}}2
	}

%%%%%%%%%%%%%%%%%%% Linguistik-Pakete %%%%%%%%%%%%%%%%%%%%%%%%
\usepackage{mathtools} % includes amsmath, adds some nice fixes
\usepackage{amssymb} % für \varnothing
%\usepackage{semantic} % für |[ |]
%\usepackage{qtree} % qtree \Tree [. ] [. ] \qroof{}.
%\usepackage[x11names, rgb]{xcolor} % für dot2tex
%\usepackage{tikz} % für dot2tex
%\usetikzlibrary{arrows,shapes} % für dot2tex
         %%% unbedingt nach tikz-Paketen laden %%%
\usepackage{gb4e} % Beispiel-Umgebung: \begin{exe} \ex \begin{xlist}
%\usepackage{avm}             % für AVMs
%\avmfont{\sc}                % allgemeine AVM-Schriftart
%\avmvalfont{\it}             % Schriftart für Werte
%\avmsortfont{\footnotesize\it} % Schriftart für Sort Labels 

%%%%%%%%%%%%%%%%%%% hyperref %%%%%%%%%%%%%%%%%%%%%%%%
\usepackage[hidelinks]{hyperref} % letzter Paketaufruf!
\makeatletter % changes the catcode of @ to 11
\AtBeginDocument{
  \hypersetup{ % hyperref: \title und \author in PDF-Eingenschaften übernehmen
    pdftitle = {\@title},
    pdfsubject = {\@subject},
    pdfauthor = {\@author}
  }
}
\makeatother % changes the catcode of @ back to 12

%%%%%%%%%%%%%%%%%%%%%%% Titel %%%%%%%%%%%%%%%%%%%%%%%%%%
\subject{Korpuslinguistische Untersuchungen zur Veränderung von Wortbedeutungen durch Medienge- und -missbrauch}
\title{Vom Wald in die Weltpolitik}
\subtitle{Die Entwicklung des Nachhaltigkeitsbegriffs \\zwischen Fach- und Allgemeinsprache}
%\setkomafont{author}{\normalsize}
\setkomafont{date}{\normalsize}
\setkomafont{publishers}{\normalsize}
\author{Sebastian Golly\\ {\normalsize (761737)}}
\date{\today}

% Double line spacing for text body, but not for titles
% TODO: Activate double line spacing
\usepackage[onehalfspacing]{setspace}
\addtokomafont{disposition}{\linespread{1}}

% No extra line spacing between list items
\usepackage{enumitem}
\setlist{nosep}

% \toprule \midrule \cmidrule \bottomrule
\usepackage{booktabs}

\usepackage{threeparttable}

% URL definitions
\usepackage{url}
\urldef{\urlcorpussize}\url{http://kaskade.dwds.de/dstar/public/hist.perl?fmt=hist&pformat=svg&q=%24l%3D%2Fnachhaltig%2Fi&_s=submit&n=date%2Bclass&smooth=none&sg=1&grid=1&sl=10%2B7&w=0&wb=0&pr=0&xr=1777%3A2016&yr=0%3A*&psize=840%2C480&T=1}
\urldef{\urlwortverlaufgesamt}\url{https://www.dwds.de/r/plot?view=3&norm=date%2Bclass&smooth=spline&genres=0&grand=1&slice=10&prune=0&window=0&wbase=0&logavg=0&logscale=0&xrange=1776%3A2016&q1=%24l%3D%2Fnachhaltig%2Fi}
\urldef{\urlwortverlaufA}\url{https://www.dwds.de/r/plot?view=3&norm=date%2Bclass&smooth=spline&genres=0&grand=1&slice=10&prune=0&window=0&wbase=0&logavg=0&logscale=0&xrange=1776%3A1845&q1=%24l%3D%2Fnachhaltig%2Fi}
\urldef{\urlwortverlaufB}\url{https://www.dwds.de/r/plot?view=3&norm=date%2Bclass&smooth=spline&genres=0&grand=1&slice=10&prune=0&window=0&wbase=0&logavg=0&logscale=0&xrange=1846%3A1915&q1=%24l%3D%2Fnachhaltig%2Fi}
\urldef{\urlwortverlaufC}\url{https://www.dwds.de/r/plot?view=3&norm=date%2Bclass&smooth=spline&genres=0&grand=1&slice=10&prune=0&window=0&wbase=0&logavg=0&logscale=0&xrange=1916%3A1985&q1=%24l%3D%2Fnachhaltig%2Fi}
\urldef{\urlwortverlaufD}\url{https://www.dwds.de/r/plot?view=3&norm=date%2Bclass&smooth=spline&genres=0&grand=1&slice=10&prune=0&window=0&wbase=0&logavg=0&logscale=0&xrange=1986%3A2016&q1=%24l%3D%2Fnachhaltig%2Fi}
\urldef{\urlkollokC}\url{http://kaskade.dwds.de/dstar/public/diacollo/?query=%24l%3D%2Fnachhaltig%2Fi&date=1916%3A1985&slice=0&score=ld&kbest=4&cutoff=&profile=2&format=html&groupby=&eps=0}
\urldef{\urlkollokD}\url{http://kaskade.dwds.de/dstar/public/diacollo/?query=%24l%3D%2Fnachhaltig%2Fi&date=1986%3A2016&slice=0&score=ld&kbest=5&cutoff=&profile=2&format=html&groupby=&eps=0}
%\urldef{\urlXXX}\url{}


\begin{document}
%%%%%%%%%%%%%%%%%%%%%%%%%%%%%%%%%%%%%%%%%%%%%%%%%%%%%%%%%%%%%%
\maketitle

\vfill

\paragraph{Abstract}

Die Benennung \textit{nachhaltig} sieht sich der Kritik ausgesetzt, zunehmend wahllos, diffus und inhaltsleer gebraucht zu werden. Die vorliegende Arbeit untersucht mittels einer diachronen korpuslinguistischen Analyse, inwieweit diese Kritik gerechtfertigt ist. Hierzu analysiert sie die Verwendungshäufigkeiten, Kontexte und Kollokationen der Benennung im Zeitraum von 1776 bis 2016. Die Ergebnisse zeigen, dass eine allgemeinsprachliche Verwendung des Nachhaltigkeitsbegriffs kein Phänomen der jüngeren Zeit, sondern bereits seit dem späten 18. Jahrhundert durchgängig nachweisbar ist.
\\[3em]

\vfill

\begin{center}
Universität Potsdam\\[1.5em]
Sommersemester 2017
\end{center}

\thispagestyle{empty}
\newpage

%%%%%%%%%%%%%%%%%%%%%%%%%%%%%%%%%%%%%%%%%%%%%%%%%%%%%%%%%%%%%%

\section{Einleitung}
\label{sec:einleitung}

Die Benennung \textit{nachhaltig} hat in den vergangenen Jahrzehnten eine beeindruckende Karriere gemacht. Beschrieb sie einst ausschließlich das forstwirtschaftliche Prinzip, nicht mehr Holz zu fällen als auch nachwachsen kann, wird sie mittlerweile in einer Vielzahl von Kontexten verwendet: „nachhaltige Entwicklung“, „nachhaltiges Wachstum“, „nachhaltige Mobilität“, „nachhaltige Erholung“, „nachhaltiger Tourismus“ und viele weitere mehr.

Bereits seit Mitte der 1990er Jahre regt sich Kritik an einer inflationären und oftmals unscharfen Verwendung des Nachhaltigkeitsbegriffs. So wird er etwa als „beliebiger Platzhalter“ \citep[141]{Vogel2011} bezeichnet, der „in den denkwürdigsten Zusammenhängen“ \citep[46]{Ninck1997} vorkommt, „mit den Ansprüchen des ursprünglichen Konzeptes und seinen wissenschaftlichen Ausformungen nichts mehr gemein hat und lediglich auf ein blasses Alltagsverständnis von Nachhaltigkeit [...] rekurriert“ \citep[5]{OhlmeierBrunold2015}.

Die vorliegende Arbeit untersucht mittels einer diachronen korpuslinguistischen Analyse, inwieweit diese Kritikpunkte gerechtfertigt sind. Zu diesem Zweck wurden Korpusbelege aus dem Zeitraum von 1776 bis 2016 gesichtet und systematisiert, Kontext- und Kollokationsanalysen durchgeführt und die Entwicklung der Verwendungshäufigkeit von \textit{nachhaltig} und verwandten Benennungen ermittelt.

Die Arbeit hat folgende Struktur: Abschnitt~\ref{sec:entwicklung-definition} skizziert, wie sich der Nachhaltigkeitsbegriff seit seinen Ursprüngen entwickelt hat, wie er heute definiert und mit welcher Kritik er konfrontiert wird. Abschnitt~\ref{sec:ziele} beschreibt die Ziele und Fragestellungen der Studie und Abschnitt~\ref{sec:methodik} die Methoden, mit denen diese Fragen beantwortet werden sollen. Abschnitt~\ref{sec:analysen} stellt die Ergebnisse der Analysen dar, wobei zunächst die Frequenzverteilung im Gesamtzeitraum und dann die Entwicklung der Frequenzen und Kontexte in vier Epochen gegliedert untersucht werden. Abschnitt~\ref{sec:diskussion} führt die Ergebnisse der Analysen auf die Fragestellungen zurück und ordnet sie mit Bezug auf die Kritik am Nachhaltigkeitsbegriff ein. Schließlich fasst Abschnitt~\ref{sec:zusammenfassung} die Studie zusammen.

\section{Der Nachhaltigkeitsbegriff}
\label{sec:entwicklung-definition}

\subsection{Entwicklung}
\label{subsec:entwicklung}

Die Ursprünge der Benennung \textit{nachhaltig} gehen auf die forstwissenschaftliche Literatur des frühen 18.~Jahrhunderts zurück. Laut \citet[99]{Zürcher1965} datiert die früheste bekannte Verwendung auf das Jahr 1713. Theodor Hartigs \textit{forstliches und forstnaturwissenschaftliches Conversations-Lexikon} von 1836 definiert: „Unter dem nachhaltigen Holzertrag wird eine dauernd, jährlich gleich große Nutzung verstanden. Der Forstbetrieb soll derart gestaltet sein, daß ein gleichmäßiger Holzanfall für alle Zukunft gesichert ist“ \citep[zitiert nach][100]{Zürcher1965}. Dementsprechend solle der Forst so bewirtschaftet werden, dass „die Nachkommenschaft wenigstens ebensoviel Vortheil daraus ziehen kann, wie sich die jetzt lebende Generation zueignet“ (\citealt[1]{Hartig1804}, zitiert nach \citealt[7]{OhlmeierBrunold2015}).

Neben der fachsprachlichen Verwendung im forstwirtschaftlichen Kontext wird \textit{nachhaltig} bereits früh allgemeinsprachlich verwendet. Jacob und Wilhelm Grimms \textit{Deutsches Wörterbuch} von 1889 umschreibt die Benennung mit „auf lange Zeit anhaltend und wirkend“ (zitiert nach \citealt[145]{Zürcher1965}). \citet[43]{Ninck1997} geht von einem Eingang des Worts in den allgemeinen deutschen Sprachgebrauch etwa um 1800 aus. Das Stilwörterbuch \textit{Duden} von 1956 bezeichnet \textit{nachhaltig} als „Papierdeutsch, Amts- und Kanzleistil“ (zitiert nach \citealt[145]{Zürcher1965}) und gibt als Verwendungsbeispiele:

\begin{itemize}
\item \textit{als Adjektiv:} eine nachhaltige (besser: fortdauernde, anhaltende) Wirkung; ein nachhaltiger (besser: wirksamer, erfolgreicher) Widerstand,
\item \textit{als Adverb:} jemanden nachhaltig (besser: wirksam, dauernd) beeinflussen.
\end{itemize}

Während diese allgemeinsprachliche Bedeutung relativ konstant zu sein scheint, erfährt die fachsprachliche Definition 1972 mit der wegweisenden Studie \textit{Die Grenzen des Wachstums} des \textit{Club of Rome} eine grundlegende Erweiterung. Eine nachhaltige Lebensweise wird hier so verstanden, „dass jede Generation ihre eigenen Probleme zu lösen hat, ohne sie den nachfolgenden Generationen aufzubürden“ \citep[8]{OhlmeierBrunold2015} – die Idee, einen Forst im Hinblick auf seine begrenzten Ressourcen und mit Rücksicht auf zukünftige Generationen schonend zu bewirtschaften, wird also übertragen und schließt nun ökonomische, ökologische und soziale Aspekte ein.

Der sogenannte \textit{Brundtland-Bericht} der Vereinten Nationen greift dieses Verständnis von Nachhaltigkeit 1987 auf und führt das Konzept der \textit{nachhaltigen Entwicklung} in den ökologischen und entwicklungspolitischen Diskurs ein. Dieses definiert er wie folgt:

\begin{quote}
„Sustainable development is development that meets the needs of the present without compromising the ability of future generations to meet their own needs“ \citep{UN1987}
\end{quote}

\begin{quote}
„Nachhaltige Entwicklung ist Entwicklung, die die Bedürfnisse der Gegenwart befriedigt, ohne zu riskieren, dass künftige Generationen ihre eigenen Bedürfnisse nicht befriedigen können“ \citep[Übersetzung durch][51]{Ninck1997}
\end{quote}

Auf dieser Basis beschließt die Weltstaatengemeinschaft auf der Konferenz der Vereinten Nationen für Umwelt und Entwicklung 1992 in Rio de Janeiro (\textit{Rio-Konferenz}) Leitlinien für das 21. Jahrhundert mit nachhaltiger Entwicklung als zentralem Ziel. Dabei soll die Idee der nachhaltigen Entwicklung in allen gesellschaftlichen Lebensbereichen verankert werden. Der Zeitraum von 2005 bis 2014 wird von den Vereinten Nationen gar zur \textit{Weltdekade der Bildung für nachhaltige Entwicklung} ausgerufen \citep[vgl.][12]{OhlmeierBrunold2015}.

Angesichts dieser Begriffsgeschichte resümiert \citet[10]{Ninck1997}: „‚Nachhaltigkeit‘, ‚Nachhaltige Entwicklung‘ sind Wörter, die es von kleinen Nobodys zu Stars der internationalen Verständigung gebracht haben.“

\subsection{Definition}
\label{subsec:definition}

Die aktuelle Ausgabe des \textit{Duden} unterscheidet zwei Lesarten von \textit{Nachhaltigkeit}:

\begin{enumerate}[rightmargin=1cm]
\item längere Zeit anhaltende Wirkung
\item 
	\begin{enumerate}
	\item (Forstwirtschaft) forstwirtschaftliches Prinzip, nach dem nicht mehr Holz gefällt werden darf, als jeweils nachwachsen kann
	\item (Ökologie) Prinzip, nach dem nicht mehr verbraucht werden darf, als jeweils nachwachsen, sich regenerieren, künftig wieder bereitgestellt werden kann
	\end{enumerate}
	\hfill\citep{Duden2017}
\end{enumerate}

Dabei unterscheidet sich die erste, allgemeinsprachliche Lesart kaum von der Definition, die bereits 1889 im Grimmschen Wörterbuch zu finden ist. Die Definition der zweiten, fachsprachlichen Lesart scheint dagegen der Spannbreite des Begriffs und der Komplexität seiner Verwendung nicht gerecht zu werden.

Bereits im Jahr 1965, also weit vor einer Thematisierung in der internationalen Politik, bemerkt Zürcher mit Blick auf den forstwissenschaftlichen Nachhaltigkeitsbegriff: „Nachhaltigkeit wird meistens sehr allgemein umschrieben. Je allgemeiner und umfassender eine Begriffsdefinition ist, um so schwieriger ist die eindeutige Festlegung ihres Inhalts“ \citep[94]{Zürcher1965}. Diese Vagheit führt mit dem politischen Aufstieg des Worts zu einer regelrechten Explosion von Definitionsansätzen. \citet[1]{Kastenholz-et-al1996} zählen „[b]ei einer Durchsicht der Literatur [...] mittlerweile über 60 unterschiedliche Definitionen von Nachhaltigkeit“.

\citet[51ff.]{Ninck1997} gibt einen umfassenden Überblick über die verschiedenen Definitionsversuche. Trotz aller unterschiedlicher Sichtweisen sieht er „so etwas wie eine stille Übereinkunft unter denen, die für das Ziel der Nachhaltigkeit wissenschaftlich gearbeitet haben oder politisch dafür eintreten, dass die Brundtland-Definition eine ‚gültige‘ Definition ist“ \citep[54]{Ninck1997}. Aus diesem Grund soll sie in dieser Arbeit als fachsprachliche Definition des Nachhaltigkeitsbegriffs dienen.

\subsection{Kritik}
\label{subsec:kritik}

Bereits seit Mitte der 1990er Jahre sieht sich der Nachhaltigkeitsbegriff bei seiner wachsenden Popularität auch wiederkehrender Kritik ausgesetzt. So stellt \citet[178]{Görg1996} fest, „dass sich die Häufigkeit der Verwendung des Begriffs umgekehrt proportional zur Bestimmtheit seines Inhalts verhalte“ \citep[zitiert nach ][140]{Vogel2011}. \citet[5]{OhlmeierBrunold2015} beobachten: „Der Terminus der Nachhaltigkeit, der u.\,a. mit universalethischen Intentionen sowie mit einer komplexen und facettenreichen Begriffsgeschichte in ursprünglich ökologischen Kontexten verbunden ist, taucht gegenwärtig in einem inflationären Ausmaß in den verschiedensten gesellschaftlichen Kontexten und Lebenszusammenhängen auf.“

Auch \citet[46]{Ninck1997} gibt zu bedenken: „Relativ neu ist, dass das Wort in ungezählten Kontexten auftaucht. So hat es unterdessen etwas Beliebiges und kommt schon in den denkwürdigsten Zusammenhängen vor.“ Für solche Kontexte geben \citet[5]{OhlmeierBrunold2015} Beispiele: „Wenn beispielsweise von einer nachhaltigen Verschönerung eines Ortszentrums, einer nachhaltigen Motivation von Mitarbeitern in Betrieben, einer nachhaltigen Entwicklung von Aktienkursen etc. die Rede ist, deutet dies zudem auf eine willkürliche Verwendung hin, die mit den Ansprüchen des ursprünglichen Konzeptes und seinen wissenschaftlichen Ausformungen nichts mehr gemein hat und lediglich auf ein blasses Alltagsverständnis von Nachhaltigkeit im Sinne einer längere Zeit anhaltenden Wirkung (Duden 2014) rekurriert.“

In Anbetracht der tiefgreifenden Auswirkungen auf Politik, Wirtschaft und Gesellschaft, die ein ernsthaftes Verständnis des Nachhaltigkeitskonzepts mit sich brächte, entsteht teilweise gar der Verdacht einer bewussten Verwässerung des Begriffs durch wirtschaftliche und politische Interessengruppen. So „dürften von der Inflationierung der Begriffsverwendung jedoch am ehesten diejenigen Kräfte profitieren, welche die ökologische Brisanz der Nachhaltigkeitsidee entschärfen wollen“ \citep[6]{OhlmeierBrunold2015}.

\section{Ziele der Studie}
\label{sec:ziele}

Wie in Abschnitt \ref{subsec:kritik} umrissen wurde, kritisieren verschiedene Wissenschaftler bereits seit den 1990er Jahren, dass die Benennung \textit{nachhaltig} so inflationär und unscharf verwendet wird, dass ihre fachsprachliche Bedeutung (etwa im Sinne der Brundtland-Definition) mehr und mehr verwässert und im Diskurs zugunsten einer blassen allgemeinsprachlichen Bedeutung (als Synonym zu \textit{dauerhaft}, \textit{längere Zeit anhaltend}) verloren geht. Allerdings gibt es meines Wissens bisher keine korpuslinguistische Untersuchung, die diese These empirisch überprüft. Die vorliegende Arbeit soll diese Lücke schließen.

Zu diesem Zweck soll sie folgende Fragstellungen untersuchen:

\begin{enumerate}[rightmargin=1cm]
\item[\textbf{F1}] Wie hat sich die Verwendungshäufigkeit der Benennung \textit{nachhaltig} seit ihrem ersten Auftreten entwickelt?
\item[\textbf{F2}] Wie haben sich in diesem Zeitraum die Kontexte verändert, in denen \textit{nachhaltig} auftritt?
\item[\textbf{F3}] Wird die Benennung \textit{nachhaltig} in verschiedenen Zeitabschnitten eher in ihrer fachsprachlichen Bedeutung (im Sinne der Brundtland-Definition), in ihrer allgemeinsprachlichen Bedeutung (als Synonym zu \textit{dauerhaft}, \textit{längere Zeit anhaltend}) oder gemischt verwendet?
\end{enumerate}

Korpuslinguistisch fundierte Antworten auf diese Fragen sollen dazu beitragen, die Diskussion zu versachlichen, und einen Impuls weg von einer wenig konstruktiven kulturpessimistisch-sprachkritischen Argumentation geben.

\section{Methodik}
\label{sec:methodik}

Die Beantwortung der Fragestellungen aus Abschnitt~\ref{sec:ziele} erfordert eine diachrone korpuslinguistische Untersuchung auf Grundlage eines Korpus, das möglichst die gesamte Zeitspanne vom ersten Auftreten der Benennung \textit{nachhaltig} bis heute abdeckt.

Es gibt nicht viele Korpora des Deutschen, die diese Anforderung erfüllen. Die Wahl fiel auf die \textit{Referenz- und Zeitungskorpora (aggregiert)} des Digitalen Wörterbuchs der deutschen Sprache (DWDS). Diese aggregierte Textsammlung vereint die folgenden Korpora:

\begin{itemize}
\item Deutsches Textarchiv (1600–1900) \citep[vgl.][]{Geyken2011}
\item DWDS-Kernkorpus (1900–1999) \citep[vgl.][]{Geyken2007}
\item DWDS-Kernkorpus 21 (2000–2010) \citep[vgl.][]{Geyken2007}
\item die Zeitungskorpora des DWDS, bestehend aus
	\begin{itemize}
	\item Berliner Zeitung (1946–2005)
	\item neues deutschland (1946–1990)
	\item Der Tagesspiegel (1996–2005)
	\item Die ZEIT (1946–2016)
	\end{itemize}
\end{itemize}

Damit deckt das aggregierte Korpus die gesamte Zeitspanne von 1600 bis 2016 mit einer großen Bandbreite verschiedener Textsorten und Disziplinen ab. Da die älteste bekannte Verwendung der Benennung \textit{nachhaltig} auf das Jahr 1713 datiert \citep[vgl.][99]{Zürcher1965}, ist das Korpus für die Zwecke dieser Studie gut geeignet.

Um verschiede Aspekte der Fragestellungen zu untersuchen, kamen unterschiedliche Werkzeuge zum Einsatz:

\begin{itemize}
\item \textit{DiaCollo}\footnote{Verfügbar unter \url{http://kaskade.dwds.de/dstar/public/diacollo/}.} \citep{Jurish2015} für Frequenz- und Kollokationsanalysen,
\item die \textit{DWDS-Wortverlaufskurve}\footnote{Verfügbar unter \url{https://www.dwds.de/r/plot}.} zur Visualisierung von Frequenzverteilungen,
\item die \textit{DWDS-Korpusabfrage}\footnote{Verfügbar unter \url{https://www.dwds.de/r}.} zur Kontextanalyse von Korpusbelegen.
\end{itemize}

Für alle Analysen wurde \lstinline|$l=/nachhaltig/i| als (Teil der) Anfrage verwendet. Sie wird durch alle Tokens erfüllt, in deren Lemma die Zeichenkette „nachhaltig“ vorkommt (unabhängig von Groß- und Kleinschreibung), also etwa „nachhaltig“, „unnachhaltig“, „nachhaltigste“, „Nachhaltigkeit“, „UN-Nach\-haltig\-keits\-ziel“ oder „Weltnachhaltigkeitsgipfel“.

Der früheste Korpusbeleg für diese Anfrage datiert auf das Jahr 1779. Um die Auswertungen in Zehnjahreszeiträume segmentieren zu können, wurde der Gesamtzeitrahmen der Analyse auf den Zeitraum von 1776 bis 2016 festgelegt. Für die detaillierten Kontext- und Kollokationsanalysen wurde er in vier Epochen untergliedert:

\begin{itemize}
\item 1776–1845 (70 Jahre)
\item 1846–1915 (70 Jahre)
\item 1916–1985 (70 Jahre)
\item 1986–2016 (31 Jahre)
\end{itemize}

Zu beachten ist allerdings, dass die Textmenge im Korpus über die Zeit nicht konstant ist. Abbildung~\ref{fig:corpus-size} zeigt die Verteilung der Anzahl von Tokens im Korpus pro Zehnjahreszeitraum seit 1776. Beim diachronen Vergleich von Auftretenshäufigkeiten einer Benennung oder Kollokation sind absolute Frequenzen deshalb kein aussagekräftiger Indikator. Stattdessen wird in dieser Arbeit vor allem die normalisierte Frequenz pro Million Tokens als Metrik verwendet.

\begin{figure}[h!]
\centering
\includegraphics[width=.8\textwidth]{img/dstar-public-corpus-size-1777-2016-edit}
\caption[corpus-size]{Anzahl von Tokens im Korpus pro Zehnjahreszeitraum seit 1776\footnotemark}
\label{fig:corpus-size}
\end{figure}


\section{Analysen}
\label{sec:analysen}

% Abbildung 1
\footnotetext{Erstellt mit D*/public Time Series: \urlcorpussize, abgerufen am 08.09.2017.}

\subsection{Frequenzanalyse im Gesamtzeitraum}
\label{subsec:freq-gesamt}

Insgesamt gibt es im Korpus 28.230 Belege für Tokens, in deren Lemma die Zeichenkette „nachhaltig“ vorkommt. Der früheste Korpusbeleg datiert auf das Jahr 1779, der späteste auf 2016.

Abbildung~\ref{fig:nachhaltig-freq-1776-2016} zeigt die normalisierten Auftretenshäufigkeiten von \textit{nachhaltig} und verwandten Wörtern im Zeitraum zwischen 1776 und 2016. Während sich die Frequenz bis 1945 im Bereich zwischen 0 und 10 Instanzen pro Million Tokens (\textit{IpM}) bewegt, steigt sie bis 1986 leicht auf etwa 17 IpM. Ab 1996 kommt es zu einem starken Anstieg der Verwendungshäufigkeit, die im Zeitraum 2006–2015 den Spitzenwert von rund 41 IpM erreicht. Im Jahr 2016 fällt die normalisierte Frequenz schließlich auf etwa 33 Vorkommen pro Million Tokens ab.

\begin{figure}[h!]
	\centering
	
	\includegraphics[width=.8\textwidth]{img/nachhaltig-freq-1776-2016}
	\caption[corpus-size]{DWDS-Wortverlaufskurve für \lstinline|$l=/nachhaltig/i| im Zeitraum 1776–2016\footnotemark}
	\label{fig:nachhaltig-freq-1776-2016}
\end{figure}

Tabelle~\ref{tab:freq-gesamt} zeigt die durchschnittlichen Frequenzen von \textit{nachhaltig} in den einzelnen Epochen dieser Untersuchung. Dabei bezeichnet $f_{abs}$ die absoluten Auftretenshäufigkeiten und $f_{norm}$ die normalisierten Frequenzen pro Million Tokens.

\begin{table}[h!]
\centering
\renewcommand{\arraystretch}{1.5}

\caption{Frequenzentwicklung von \textit{nachhaltig} in den untersuchten Epochen}
\label{tab:freq-gesamt}

\begin{threeparttable}

\begin{tabular}{ccc}
\textbf{Epoche} & \boldmath{$f_{abs}$} & \boldmath{$f_{norm}$} \\ \hline
1776–1845 & 206 & 3,70 \\ \hline
1846–1915 & 597 & 6,89 \\ \hline
1916–1985 & 3.071 & 12,91 \\ \hline
1986–2016 & 24.356 & 29,63 \\ \hline\hline
1776–2016 & 28.230 & 23,48 \\ \hline
\end{tabular} 

\begin{tablenotes}
\footnotesize
\setlength{\itemindent}{-1.2em}
\item $f_{abs}$: absolute Frequenz
\item $f_{norm}$: normalisierte Frequenz pro Million Tokens (IpM)
\end{tablenotes}

\end{threeparttable}
\end{table}
	
Der Frequenzverlauf belegt einen deutlichen Anstieg der Verwendungshäufigkeit von \textit{nachhaltig} seit Mitte der 1990er Jahre. Ob diese quantitative Entwicklung mit einer Bedeutungsverschiebung einhergeht, soll im Folgenden eine detaillierte Analyse der Kontexte und Kollokationen in den einzelnen Epochen zeigen.

\subsection{Detailanalyse 1776–1845}
\label{subsec:detail-1776–1845}

% Abbildung 2
\footnotetext{Erstellt durch das Digitale Wörterbuch der deutschen Sprache: \urlwortverlaufgesamt, abgerufen am 05.09.2017.}

\subsubsection{Frequenzanalyse}

In der gesamten Epoche von 1776 bis 1845 gibt es nur 206 Korpusbelege für \textit{nachhaltig} und verwandte Benennungen, was durchschnittlich 3,7 Instanzen pro Million Tokens entspricht. Tabelle~\ref{tab:freq-epoche1} zeigt die Frequenzentwicklung pro Zehnjahreszeitraum, Abbildung~\ref{fig:nachhaltig-freq-1776-1845} illustriert diese Entwicklung.

\begin{table}[h!]
	\centering
	\renewcommand{\arraystretch}{1.5}
	
	\caption{Frequenzentwicklung von \textit{nachhaltig} in der Epoche 1776–1845}
	\label{tab:freq-epoche1}
	
	\begin{threeparttable}
	
	\begin{tabular}{ccc}
	\textbf{Zeitraum} & \boldmath{$f_{abs}$} & \boldmath{$f_{norm}$} \\ \hline
	1776–1785 & 2 & 0,24 \\ \hline
	1786–1795 & 0 & 0,00 \\ \hline
	1796–1805 & 2 & 0,23 \\ \hline
	1806–1815 & 58 & 9,04 \\ \hline
	1816–1825 & 4 & 0,86 \\ \hline
	1826–1835 & 42 & 4,89 \\ \hline
	1836–1845 & 98 & 8,89 \\ \hline\hline
	1776–1845 & 206 & 3,70 \\ \hline
	\end{tabular} 
	
	\begin{tablenotes}
	\footnotesize
	\setlength{\itemindent}{-1.2em}
	\item $f_{abs}$: absolute Frequenz
	\item $f_{norm}$: normalisierte Frequenz pro Million Tokens (IpM)
	\end{tablenotes}
	
	\end{threeparttable}
\end{table}

\begin{figure}[h!]
	\centering
	
	\includegraphics[width=.8\textwidth]{img/nachhaltig-freq-1776-1845}
	\caption[corpus-size]{DWDS-Wortverlaufskurve für \lstinline|$l=/nachhaltig/i| im Zeitraum 1776–1845\protect\footnotemark}
	\label{fig:nachhaltig-freq-1776-1845}
\end{figure}

Die beobachteten Gebrauchshäufigkeiten sind in diesem Zeitraum so niedrig, dass die beiden Maxima in der Frequenzverteilung auf einzelne, umfangreiche Werke zurückzuführen sind, in denen „nachhaltig“ sehr häufig verwendet wird. Dabei handelt es sich um \textit{Grundsätze der rationellen Landwirthschaft} von Albrecht Thaer (erschienen 1809) und \textit{Die römischen Päpste} von Leopold Ranke (erschienen 1836).

% Abbildung 3
\footnotetext{Erstellt durch das Digitale Wörterbuch der deutschen Sprache: \urlwortverlaufA, abgerufen am 05.09.2017.}

Die Frequenzen sind auch zu niedrig, um eine aussagekräftige Kollokationsanalyse durchzuführen. Stattdessen sollen im Folgenden die Kontexte untersucht werden, in denen \textit{nachhaltig} in dieser Epoche auftritt.

\subsubsection{Kontextanalyse}

Die bei Weitem meisten Vorkommen von \textit{nachhaltig} im Zeitraum 1776–1845 zeigen eine allgemeinsprachliche Verwendung in den Bedeutungen „dauerhaft“ oder „lange Zeit anhaltend“. Interessant ist dabei, dass es erste Belege für einen Übergang in die Allgemeinsprache bereits deutlich vor dem Jahr 1800 gibt. Die Korpusbelege~\ref{ex-1776-1845-allg-1}–\ref{ex-1776-1845-allg-n} führen einige typische Beispiele für eine allgemeinsprachliche Verwendung von \textit{nachhaltig} auf.

\begin{exe}
\ex \label{ex-1776-1845-allg-1} „Der ordentliche Zehrpfennig reichte freilich nicht weit; aber der Spar- und der Nothpfennig waren desto nachhaltiger, und ließen uns unterwegens nicht darben.“ (Musäus, Johann Karl August: Physiognomische Reisen. Bd. 4. Altenburg, 1779.)
\ex „Wilhelm hatte sich in diesem Falle befunden, er schien nunmehr zum erstenmal zu merken, daß er äußerer Hülfsmittel bedürfe, um nachhaltig zu wirken.“ (Goethe, Johann Wolfgang von: Wilhelm Meisters Lehrjahre. Bd. 4. Frankfurt (Main) u. a., 1796.)
\ex „die Anstrengung eines ernsten nachhaltigen Denkens“ (Allgemeine Zeitung. Beilage zu Nr. 106. Stuttgart, 15. April 1804.)
\ex „nachhaltige Verstimmung“ (Jean Paul: Dritte Abteilung Briefe. 1822. In: Jean Pauls Sämtliche Werke. Historisch-kritische Ausgabe. Abt. 3, Bd. 8. Berlin, 1955.)
\ex „nachhaltiges Interesse“ (Rumohr, Karl Friedrich von: Italienische Forschungen. T. 3. Berlin u. a., 1831.)
\ex „eine stille, sehr nachhaltige Neigung“ (Mörike, Eduard: Maler Nolten. Bd. 2 Stuttgart, 1832.)
\ex „eine nachhaltige gewaltige Hülfe“ (Ranke, Leopold von: Die römischen Päpste. Bd. 1. Berlin, 1834.)
\ex „nachhaltiger Eindruck“ (Clausewitz, Carl von: Vom Kriege. Bd. 3. Berlin, 1834.)
\ex „nachhaltiges Vertrauen“ (Ranke, Leopold von: Die römischen Päpste. Bd. 2. Berlin, 1836.)
\ex „nachhaltigen Widerstand“ (Ranke, Leopold von: Die römischen Päpste. Bd. 2. Berlin, 1836.)
\ex „nachhaltiger Herzstärkung“ (Varnhagen von Ense, Karl August: Denkwürdigkeiten und vermischte Schriften. Bd. 2. Mannheim, 1837.)
\ex „nachhaltigen Humor und Eifer“ (Varnhagen von Ense, Karl August: Denkwürdigkeiten und vermischte Schriften. Bd. 2. Mannheim, 1837.)
\ex „daß der Alte [...] erst zwar nachhaltig den Kopf schüttelte“ (Immermann, Karl: Münchhausen. Bd. 1. Düsseldorf, 1838.)
\ex „der einzige Ausgangspunkt nachhaltiger Verbesserungen“ (Allgemeine Zeitung. Beilage zu Nr. 17. Stuttgart, 17. Januar 1840.)
\ex „ein dichtes nachhaltiges Feldgeschrei“ (Allgemeine Zeitung. Beilage zu Nr. 92. 01.04.1840.)
\ex \label{ex-1776-1845-allg-n} „Auch ergiebt sich das Talent nur Solchen, welche etwas nachhaltig wollen.“ (Dahlmann, Friedrich Christoph: Geschichte der französischen Revolution bis auf die Stiftung der Republik. Leipzig, 1845.)
\end{exe}

Eine Verwendung in der ursprünglichen fachwissenschaftlichen Bedeutung im Kontext der Forstwirtschaft kann dagegen nur vereinzelt nachgewiesen werden, wie in den Korpusbelegen~\ref{Baumstark1835forst} und \ref{Baumstark1835forst2}.

\begin{exe}
\ex \label{Baumstark1835forst} „Die weitere Pflege der Holzpflanzen (§. 151.) hat zum Zwecke, in der kürzesten Zeit mit den geringsten Kosten, ohne die Waldwirthschaft zu zerstören, den größten Naturalertrag aus denselben zu beziehen und den Wald nachhaltig zu machen.“ (Baumstark, Eduard: Kameralistische Encyclopädie. Heidelberg u. a., 1835.)
\ex \label{Baumstark1835forst2} „Ohne Forstgründe in großer Flächenausdehnung ist ein nachhaltiger, das nöthige Holzquantum sichernder, Betrieb des Waldbaues nicht möglich“ (Baumstark, Eduard: Kameralistische Encyclopädie. Heidelberg u. a., 1835.)
\end{exe}

Bemerkenswert ist außerdem, dass bereits im Jahr 1835 eine erste Übertragung des Nachhaltigkeitsbegriffs auf andere natürliche Ressourcen zu beobachten ist, wie Beispiel~\ref{Baumstark1835jagd} zeigt.

\begin{exe}
\ex \label{Baumstark1835jagd} „daß die Jagd nachhaltig, d. h. ohne daß sie mit dem Wildstande eingeht, betrieben und benutzt werden kann“ (Baumstark, Eduard: Kameralistische Encyclopädie. Heidelberg u. a., 1835.)
\end{exe}

Gleichzeitig wird selbst im landwirtschaftlichen Kontext \textit{nachhaltig} wiederholt in seiner allgemeinsprachlichen Bedeutung verwendet, insbesondere in Bezug auf betriebswirtschaftlich-finanzielle Aspekte (Belege~\ref{Thaer1810bwl}–\ref{Baumstark1835bwl}) oder die Wirksamkeit von Düngemitteln (Belege~\ref{Thaer1810dung}–\ref{Baumstark1835dung}):

\begin{exe}
\ex \label{Thaer1810bwl} „nachhaltigen Vermehrung der Produktion“ (Thaer, Albrecht: Grundsätze der rationellen Landwirthschaft. Bd. 1. Berlin, 1809.)
\ex „wenn sich der Ertrag nach einer Reihe von Jahren nachhaltig vergrößert.“ (Thaer, Albrecht: Grundsätze der rationellen Landwirthschaft. Bd. 1. Berlin, 1809.)
\ex \label{Baumstark1835bwl} „Man muß daher als ordentlicher Wirth suchen [...] den Erwerb so sicher und dauerhaft als möglich zu erhalten, d. h. die Wirthschaft nachhaltig einzurichten und zu führen“ (Baumstark, Eduard: Kameralistische Encyclopädie. Heidelberg u. a., 1835.)
\ex \label{Thaer1810dung} „und somit strohigen Dünger macht, der zwar keine so schnelle Wirkung wie der Pferch äußert, aber ungleich nachhaltiger ist“ (Thaer, Albrecht: Grundsätze der rationellen Landwirthschaft. Bd. 2. Berlin, 1810.)
\ex \label{Baumstark1835dung} „um den Dünger nachhaltiger zu machen“ (Baumstark, Eduard: Kameralistische Encyclopädie. Heidelberg u. a., 1835.)
\end{exe}

Zusammenfassend ergibt die Kontextanalyse, dass die Benennung \textit{nachhaltig} bereits von 1776 bis 1845 vorwiegend in ihrer allgemeinsprachlichen Bedeutung verwendet wird, teilweise selbst in Fachpublikationen der Land- und Forstwirtschaft. Dabei zeigen die verschiedenen Belege aus der \textit{Kameralistischen Encyclopädie}, wie die Benennung in einer einzigen Publikation sowohl fach- als auch allgemeinsprachlich gebraucht wird.



\subsection{Detailanalyse 1846–1915}
\label{subsec:detail-1846–1915}

\subsubsection{Frequenzanalyse}

In der Epoche von 1846 bis 1915 lassen sich 597 Korpusbelege für \textit{nachhaltig} und verwandte Benennungen nachweisen. Dies entspricht einer normalisierten Frequenz von 6,89 Instanzen pro Million Tokens. Tabelle~\ref{tab:freq-epoche2} stellt die Frequenzverteilung pro Zehnjahreszeitraum dar, Abbildung~\ref{fig:nachhaltig-freq-1846-1915} illustriert sie.

\begin{table}[h!]
	\centering
	\renewcommand{\arraystretch}{1.5}
	
	\caption{Frequenzentwicklung von \textit{nachhaltig} in der Epoche 1846–1915}
	\label{tab:freq-epoche2}
	
	\begin{threeparttable}
	
	\begin{tabular}{ccc}
	\textbf{Zeitraum} & \boldmath{$f_{abs}$} & \boldmath{$f_{norm}$} \\ \hline
	1846–1855 & 56 & 3,75 \\ \hline
	1856–1865 & 65 & 6,80 \\ \hline
	1866–1875 & 70 & 7,41 \\ \hline
	1876–1885 & 61 & 6,92 \\ \hline
	1886–1895 & 108 & 10,51 \\ \hline
	1896–1905 & 82 & 6,30 \\ \hline
	1906–1915 & 155 & 7,51 \\ \hline\hline
	1846–1915 & 597 & 6,89 \\ \hline
	\end{tabular} 
	
	\begin{tablenotes}
	\footnotesize
	\setlength{\itemindent}{-1.2em}
	\item $f_{abs}$: absolute Frequenz
	\item $f_{norm}$: normalisierte Frequenz pro Million Tokens (IpM)
	\end{tablenotes}
	
	\end{threeparttable}
\end{table}

\begin{figure}[h!]
	\centering
	
	\includegraphics[width=.8\textwidth]{img/nachhaltig-freq-1846-1915}
	\caption[corpus-size]{DWDS-Wortverlaufskurve für \lstinline|$l=/nachhaltig/i| im Zeitraum 1846–1915\protect\footnotemark}
	\label{fig:nachhaltig-freq-1846-1915}
\end{figure}
	\footnotetext{Erstellt durch das Digitale Wörterbuch der deutschen Sprache: \urlwortverlaufB, abgerufen am 05.09.2017.}

\subsubsection{Kontextanalyse}

Auch in dieser Epoche tritt \textit{nachhaltig} vor allem in seiner allgemeinsprachlichen Bedeutung auf. Die Korpusbelege~\ref{ex-1846-1915-allg-first}–\ref{ex-1846-1915-allg-last} illustrieren dies durch typische Verwendungsbeispiele.

\begin{exe}
\ex \label{ex-1846-1915-allg-first} „einen nachhaltigen Eindruck“ (Schoppe, Amalie: Der Prophet. Bd. 2. Jena, 1846.)
\ex „nachhaltiges Interesse“ (Heckert, Adolph (Hrsg.): Handbuch der Schulgesetzgebung Preußens. Berlin, 1847.)
\ex „nachhaltiger Beifall“ (Neue Rheinische Zeitung. Nr. 129. Köln, 29. Oktober 1848. Zweite Ausgabe.)
\ex „von nachhaltiger wissenschaftlicher Bedeutung“ (Jahn, Otto: Gottfried Herrmann. Eine Gedächnissrede. Leipzig, 1849.)
\ex „ihn [den Haß gegen die Juden] nachhaltig unschädlich zu machen“ (Die Bayerische Presse. Nr. 140. Würzburg, 12. Juni 1850.)
\ex „ein reines und nachhaltiges Vergnügen“ (Keller, Gottfried: Der grüne Heinrich. Bd. 3. Braunschweig, 1854.)
\ex „Es wird aber der Ruhm des einen von kurzer Dauer, der des andern nachhaltig sein“ (Krane, Friedrich von: Die Dressur des Reitpferdes (Campagne- und Gebrauchs-Pferdes). Münster, 1856.)
\ex „Auch kleinere Glocken werden in Japan mit hölzernem Hammer geschlagen, niemals mit metallenem Klöpfel; der so erzeugte Klang ist nicht so hell, aber voller, weicher und nachhaltiger als bei unseren Glocken.“ (Berg, Albert: Die preussische Expedition nach Ost-Asien. Bd. 1. Berlin, 1864.)
\ex „nachhaltige Wirkung“ (Carus, Julius Victor: Geschichte der Zoologie bis auf Johannes Müller und Charles Darwin. München, 1872.)
\ex „nachhaltige Unterstützung“ (Bluntschli, Johann Caspar: Allgemeine Statslehre. Stuttgart, 1875.)
\ex „von nachhaltiger Bedeutung“ (Fontane, Theodor: Wanderungen durch die Mark Brandenburg. Bd. 4: Spreeland. Berlin, 1882.)
\ex „Da mußte doch eine tiefe, nachhaltige Sympathie vorhanden sein“ (Conradi, Hermann: Adam Mensch. Leipzig, [1889].)
\ex „Eine nachhaltige Belebung der Industrie“ (Vossische Zeitung (Abend-Ausgabe), 03.03.1909)
\ex „Mir ist aus jenem Leben, so anstrengend es auch war, viel nachhaltige Freude erwachsen.“ (Bismarck, Hedwig von: Erinnerungen aus dem Leben einer 95jährigen. In: Simons, Oliver (Hg.) Deutsche Autobiographien 1690-1930, Berlin: Directmedia Publ. 2004 [1910], S. 7630)
\ex \label{ex-1846-1915-allg-last} „wo das geistige Leben der Musenstadt ihn auf das nachhaltigste beeinflußte“ (Brümmer, Franz: Lexikon der deutschen Dichter und Prosaisten vom Beginn des 19. Jahrhunderts bis zur Gegenwart. Bd. 5. 6. Aufl. Leipzig, 1913.)
\end{exe}

Innerhalb des allgemeinsprachlichen Gebrauchs wird \textit{nachhaltig} in dieser Zeit auch in neuen Bedeutungsvarianten verwendet, etwa als Synonym zu \textit{permanent}, \textit{fest} (in Bezug auf Arbeit oder Einkommen) (vgl. \ref{ex-1846-1915-permanent-1}–\ref{ex-1846-1915-permanent-2}) oder im Sinne von \textit{ausdauernd} (vgl. \ref{ex-1846-1915-ausdauernd-1}–\ref{ex-1846-1915-ausdauernd-n}).

\begin{exe}
\ex \label{ex-1846-1915-permanent-1} „wegen des auskömmlichen und nachhaltigen Verdienstes“ (Lassalle, Ferdinand: Die indirekte Steuer und die Lage der arbeitenden Klassen. Zürich, 1863.)
\ex \label{ex-1846-1915-permanent-2} „Auf nachhaltige Arbeit hatte er sich nie verstanden“ (Treitschke, Heinrich von: Deutsche Geschichte im neunzehnten Jahrhundert. Bd. 3: Bis zur Juli-Revolution. Leipzig, 1885.)
\ex \label{ex-1846-1915-ausdauernd-1} „Da zeigt sich eben wieder die Kraft und Konsequenz des Aelplers, -- der Ernst und die Ausdauer, der feste Wille und der Muth, nicht nur in Dingen des alltäglichen Müssens und Sollens, sondern auch in Sachen eigenen Entschlusses, eigener freier Meinung: so zäh wie er in seinen physischen Anstrengungen ist, ebenso nachhaltig ist er auch in den Resultaten seines Nachdenkens, seiner Willensfreiheit.“ (Berlepsch, Hermann Alexander: Die Alpen in Natur- und Lebensbildern. Leipzig, 1871.)
\ex „Auch die Armen an Geist, an Vorkenntnissen und an Geld können ihnen Angemessenes und Erreichbares ermitteln [...], wenn sie nur Willenskraft und Nachhaltigkeit besitzen.“ (Michelis, Arthur: Reiseschule für Touristen und Curgäste. Leipzig, 1869.)
\ex „Was den Frauen fehlt, ist ja nach der Männer Urtheil eben die nachhaltige Tüchtigkeit.“ (Lewald, Fanny: Für und wider die Frauen. Berlin, 1870.)
\ex „so werde das preußische Heer sich ohne Zweifel vollzähliger, kampfwilliger, schlagfertiger und nachhaltiger zeigen als das russische“ (Treitschke, Heinrich von: Deutsche Geschichte im neunzehnten Jahrhundert. Bd. 4: Bis zum Tode König Friedrich Wilhelms III. Leipzig, 1889.)
\ex „ihr fehlte die Nachhaltigkeit, und alle guten Anwandlungen gingen wieder vorüber“ (Fontane, Theodor: Effi Briest. Berlin, 1896.)
\ex \label{ex-1846-1915-ausdauernd-n} „Dem Staatssekretär gebührt unsere volle Anerkennung für den Eifer und die Nachhaltigkeit, mit der er den auf dem Rechtsgebiete nothwendigen Reformen vorarbeitet.“ (Vossische Zeitung (Abend-Ausgabe), 05.03.1903)
\end{exe}

Daneben tritt \textit{nachhaltig} in diesem Zeitraum vermehrt im forstwirtschaftlichen Kontext auf. \ref{ex-1846-1915-fach-1}–\ref{ex-1846-1915-fach-n} zeigen einige Beispiele mit eindeutig fachsprachlicher Bedeutung.

\begin{exe}
\ex \label{ex-1846-1915-fach-1} „Ausserdem erfordert die nachhaltige Holzproduktion selbst einen Vorrath lebendigen Holzes, welcher das zehn- bis vierzigfache der jährlichen Nutzung beträgt.“ (Kirchhof, Friedrich: Handbuch der landwirthschaftlichen Betriebslehre. Dessau, 1852. Zitiert nach: Marx, Karl: Das Kapital. Bd. 2. Buch II: Der Cirkulationsprocess des Kapitals. Hamburg, 1885.)
\ex „Die Forstwirtschaft in den Staatswaldungen hat die Nachhaltigkeit der Nutzung als obersten Grundsatz zu befolgen und ihren Wirtschaftsplan auf sorgfältige Ertragsermittelungen zu stützen.“ (Bayerisches Forstgesetz von 1852, Art. 2. Zitiert nach Schwappach, Adam: Forstpolitik, Jagd- und Fischereipolitik. Leipzig, 1894.)
\ex „bei nachhaltiger und pfleglicher Bewirthschaftung der Waldungen, welche aus dem Walde jährlich nicht mehr an Holzmasse hinwegnimmt, als jährlich am stehen bleibenden Holze zuwächst.“ (Roßmäßler, Emil Adolf: Der Wald. Leipzig u. a., 1863.)
\ex „Wie die in Oesterreich gemachten Erfahrungen beweisen , sind die Aktiengesellschaften zwar sehr geschickt, den Wald zu exploitieren, aber eine nachhaltige, konservative Forstwirtschaft ist ihrem Wesen fremd.“ (Schwappach, Adam: Forstpolitik, Jagd- und Fischereipolitik. Leipzig, 1894.)
\ex „Für ihre Gewinnung gilt im allgemeinen der Grundsatz, dass hierdurch die Nachhaltigkeit der Holzproduktion nicht beeinträchtigt werden dürfe.“ (Schwappach, Adam: Forstpolitik, Jagd- und Fischereipolitik. Leipzig, 1894.)
\ex \label{ex-1846-1915-fach-n} „Anderseits sollen aber die Betriebspläne unter Zugrundelegung der berechtigten Wünsche und vorhandenen Bedürfnisse der Nutzniesser angefertigt werden, ohne den Genuss der gegenwärtigen Generation weiter zu schmälern, als es die Rücksicht auf die Nachhaltigkeit erfordert.“ (Schwappach, Adam: Forstpolitik, Jagd- und Fischereipolitik. Leipzig, 1894.)
\end{exe}

Wie bereits in der vorangehenden Epoche wird der Begriff jedoch auch im forstwirtschaftlichen Kontext in seiner allgemeinsprachlichen Bedeutung verwendet, wie \ref{ex-1846-1915-forst-allg-1}–\ref{ex-1846-1915-forst-allg-n} belegen.

\begin{exe}
\ex \label{ex-1846-1915-forst-allg-1} „die nachhaltige Sicherheit des Ertrags“ (Roßmäßler, Emil Adolf: Der Wald. Leipzig u. a., 1863.)
\ex „Da die Verwesung des Düngers in dem geschlossenen, feuchten Thonboden nur langsam vor sich geht, aber um so nachhaltiger wirkt, so düngt man seltener aber mit um so größeren Mengen auf einmal.“ (Krafft, Guido: Lehrbuch der Landwirthschaft auf wissenschaftlicher und praktischer Grundlage. Bd. 1. Berlin, 1875.)
\ex „eine nachhaltige Steigerung der Erträge“ (Krafft, Guido: Lehrbuch der Landwirthschaft auf wissenschaftlicher und praktischer Grundlage. Bd. 1. Berlin, 1875.)
\ex „nachhaltige Verdauungsstörungen“ (Krafft, Guido: Lehrbuch der Landwirthschaft auf wissenschaftlicher und praktischer Grundlage. Bd. 3. Berlin, 1876.)
\ex \label{ex-1846-1915-forst-allg-n} „es schicken daher die Waldungen den Flüssen im Sommerhalbjahre einen nachhaltigeren Tribut zu, als unter sonst gleichen Umständen das freie Feld“ (Schwappach, Adam: Forstpolitik, Jagd- und Fischereipolitik. Leipzig, 1894.)
\end{exe}

Die Beispiele aus den Werken \textit{Der Wald} und \textit{Forstpolitik, Jagd- und Fischereipolitik} zeigen wieder eine gemischte Verwendung von \textit{nachhaltig} in fach- und allgemeinsprachlicher Bedeutung innerhalb einzelner Publikationen. Nicht immer lässt sich die so entstehende Mehrdeutigkeit anhand des unmittelbaren Kontexts einfach auflösen.

Insgesamt veranschaulicht die Kontextanalyse, dass \textit{nachhaltig} im Zeitraum 1846–1915 zunehmend in einer eindeutig fachsprachlichen Bedeutung gebraucht wird. Dennoch überwiegt die allgemeinsprachliche Verwendung, die um zusätzliche Kontexte und damit Bedeutungsvarianten erweitert wird. Auch in dieser Epoche ist das Phänomen einer gemischten fach- und allgemeinsprachlichen Verwendung in Fachpublikationen zu beobachten.


\subsection{Detailanalyse 1916–1985}
\label{subsec:detail-1916–1985}

\subsubsection{Frequenzanalyse}

Für die Epoche von 1916 bis 1985 enthält das Korpus 3.071 Belege für \textit{nachhaltig} und verwandte Benennungen, was einer normalisierten Frequenz von 12,91 Instanzen pro Million Tokens entspricht. Tabelle~\ref{tab:freq-epoche3} zeigt die Frequenzentwicklung pro Zehnjahreszeitraum, Abbildung~\ref{fig:nachhaltig-freq-1916-1985} illustriert diese Verteilung.

\begin{table}[h!]
	\centering
	\renewcommand{\arraystretch}{1.5}
	
	\caption{Frequenzentwicklung von \textit{nachhaltig} in der Epoche 1916–1985}
	\label{tab:freq-epoche3}
	
	\begin{threeparttable}
	
	\begin{tabular}{ccc}
	\textbf{Zeitraum} & \boldmath{$f_{abs}$} & \boldmath{$f_{norm}$} \\ \hline
	1916–1925 & 43 & 3,61 \\ \hline
	1926–1935 & 84 & 5,57 \\ \hline
	1936–1945 & 99 & 8,40 \\ \hline
	1946–1955 & 379 & 11,52 \\ \hline
	1956–1965 & 507 & 11,22 \\ \hline
	1966–1975 & 859 & 15,02 \\ \hline
	1976–1985 & 1.100 & 17,22 \\ \hline\hline
	1916–1985 & 3.071 & 12,91 \\ \hline
	\end{tabular} 
	
	\begin{tablenotes}
	\footnotesize
	\setlength{\itemindent}{-1.2em}
	\item $f_{abs}$: absolute Frequenz
	\item $f_{norm}$: normalisierte Frequenz pro Million Tokens (IpM)
	\end{tablenotes}
	
	\end{threeparttable}
\end{table}

\begin{figure}[h!]
	\centering
	
	\includegraphics[width=.8\textwidth]{img/nachhaltig-freq-1916-1985}
	\caption[corpus-size]{DWDS-Wortverlaufskurve für \lstinline|$l=/nachhaltig/i| im Zeitraum 1916–1985\protect\footnotemark}
	\label{fig:nachhaltig-freq-1916-1985}
\end{figure}

% Abbildung 5
\footnotetext{Erstellt durch das Digitale Wörterbuch der deutschen Sprache: \urlwortverlaufC, abgerufen am 05.09.2017.}

Im Vergleich zur vorangehenden Epoche startet die Frequenzverteilung mit einer relativ niedrigen Gebrauchshäufigkeit, die im Verlauf der Epoche jedoch nahezu stetig zunimmt, bis sie im Zeitraum 1976–1985 ihr bisheriges Maximum von rund 17 Instanzen pro Million Tokens erreicht.

\subsubsection{Kontextanalyse}

\textit{Nachhaltig} wird also zunehmend häufig verwendet, doch in welchen Kontexten? In dieser Epoche sind die Verwendungsfrequenzen erstmals hoch genug, um uns dieser Fragestellung mittels einer Kollokationsanalyse nähern zu können. Tabelle~\ref{tab:kollokationen-epoche3} zeigt die vier am stärksten mit \textit{nachhaltig} assoziierten Kollokationen im betrachteten Zeitraum.

\begin{table}[h!]
	\centering
	\renewcommand{\arraystretch}{1.5}
	
	\caption{Top-Kollokationen von \lstinline|$l=/nachhaltig/i| in der Epoche 1916–1985\protect\footnotemark}
	\label{tab:kollokationen-epoche3}
	
	\begin{threeparttable}
	
	\begin{tabular}{cccc}
	\textbf{Rang} & \textbf{Lemma} & \boldmath{$f_{12}$} & \textbf{logDice} \\ \hline
	1 & Zinssenkung & 9 & 3,47 \\ \hline
	2 & Eindruck & 56 & 3,30 \\ \hline
	3 & Wirkung & 50 & 3,16 \\ \hline
	4 & Forstwirtschaft & 4 & 2,32 \\ \hline
	\end{tabular} 
	
	\begin{tablenotes}
	\footnotesize
	\setlength{\itemindent}{-1.2em}
	\item $f_{12}$: absolute Kookkurenzfrequenz
	\item logDice: normalisiertes Assoziationsmaß
	\end{tablenotes}
	
	\end{threeparttable}
\end{table}

% Tabelle 5
	\footnotetext{Erstellt mit DiaCollo: \urlkollokC, abgerufen am 12.09.2017.}

Die Korpusbelege \ref{ex:epoche3-koll-1}–\ref{ex:epoche3-koll-n} geben eine Übersicht über typische Kontexte für die vier Top-Kollokationen.

\begin{enumerate}
\item Zinssenkung
	\begin{exe}
	\ex \label{ex:epoche3-koll-1} „Unter Berücksichtigung dieser Aussichten sind die Möglichkeiten einer nachhaltigen Zinssenkung gering.“ (Die Zeit, 12.10.1962, Nr. 41)
	\end{exe}
\item Eindruck
	\begin{exe}
	\ex „Unter dem nachhaltigen Eindruck der Ereignisse von 1849 rückt Marx noch entschiedener von der ‚politischen Intervention‘ ab.“ (Ball, Hugo: Zur Kritik der deutschen Intelligenz. Bern, 1919.)
	\ex „Die unmittelbarsten und nachhaltigsten Eindrücke von Schliemanns Leben und seinen Reisen in die griechische Vergangenheit sind immer noch aus Schliemanns Schriften selber zu gewinnen.“ (Die Zeit, 26.05.1972, Nr. 21)
	\end{exe}
\item Wirkung
	\begin{exe}
	\ex „die nachhaltige und einwandfreie Wirkung unserer Schutzpockenimpfung“ (Berliner Tageblatt (Abend-Ausgabe), 05.03.1917)
	\ex „Um nachhaltigere Wirkungen zu erzielen, bedurfte es anderer Mittel.“ (Die Zeit, 28.02.1964, Nr. 09)
	\end{exe}
\item Forstwirtschaft
	\begin{exe}
	\ex „Erst ganz allmählich setzte sich [...] eine geordnete, auf dem Prinzip der Nachhaltigkeit beruhende Forstwirtschaft auch in den USA durch.“ (Sandermann, Wilhelm: Grundlagen der Chemie und chemischen Technologie des Holzes, Leipzig: Geest \& Portig 1956, S. 13)
	\ex \label{ex:epoche3-koll-n} „Wichtigster Grundsatz der modernen Forstwirtschaft wurde die ‚Nachhaltigkeit‘“ (Die Zeit, 19.10.1984, Nr. 43)
	\end{exe}
\end{enumerate}

Mit Ausnahme des forstwirtschaftlichen Kontexts wird \textit{nachhaltig} hier nahezu ausschließlich in seiner allgemeinsprachlichen Bedeutung verwendet. Bemerkenswert ist, dass die Kombination „nachhaltige Entwicklung“ in dieser Epoche, die kurz vor Erscheinen des Brundtland-Berichts endet, im Korpus noch kein einziges Mal vorkommt.

Die Kontextanalyse weist also darauf hin, dass auch im Zeitraum 1916–1985 \textit{nachhaltig} in erster Linie allgemeinsprachlich gebraucht wird und nur vereinzelt im forstwirtschaftlichen Kontext in seiner fachsprachlichen Bedeutung. Auch wenn die Benennung zunehmend häufig zum Einsatz kommt, deutet sich die Übertragung des Nachhaltigkeitsbegriffs auf Ökonomie und Gesellschaft noch nicht an.


\subsection{Detailanalyse 1986–2016}
\label{subsec:detail-1986–2016}

\subsubsection{Frequenzanalyse}

In der mit 31 Jahren kürzesten Epoche von 1986 bis 2016 lassen sich 24.356 Korpusbelege für \textit{nachhaltig} und verwandte Benennungen ermitteln. Das bedeutet, dass in diese 13~Prozent des Gesamtzeitraums 86~Prozent aller Vorkommen im Korpus fallen. Dies lässt sich nur zum Teil damit erklären, dass das Korpus in diesem Zeitraum besonders umfangreich ist (vgl. Abbildung~\ref{fig:corpus-size} auf Seite~\pageref{fig:corpus-size}), wie die hohe normalisierte Gebrauchshäufigkeit von knapp 30 Instanzen pro Million Tokens zeigt. Tabelle~\ref{tab:freq-epoche4} stellt die Frequenzentwicklung pro Zehnjahreszeitraum dar, Abbildung~\ref{fig:nachhaltig-freq-1986-2016} illustriert diese Verteilung.

\begin{table}[h!]
	\centering
	\renewcommand{\arraystretch}{1.5}
	
	\caption{Frequenzentwicklung von \textit{nachhaltig} in der Epoche 1986–2016}
	\label{tab:freq-epoche4}
	
	\begin{threeparttable}
	
	\begin{tabular}{ccc}
	\textbf{Zeitraum} & \boldmath{$f_{abs}$} & \boldmath{$f_{norm}$} \\ \hline
	1986–1995 & 1.831 & 17,28 \\ \hline
	1996–2005 & 11.839 & 26,28 \\ \hline
	2006–2015 & 9.748 & 41,04 \\ \hline
	2016 & 938 & 33,48 \\ \hline
	1986–2016 & 24.356 & 29,63 \\ \hline
	\end{tabular} 
	
	\begin{tablenotes}
	\footnotesize
	\setlength{\itemindent}{-1.2em}
	\item $f_{abs}$: absolute Frequenz
	\item $f_{norm}$: normalisierte Frequenz pro Million Tokens (IpM)
	\end{tablenotes}
	
	\end{threeparttable}
\end{table}

\begin{figure}[h!]
	\centering
	
	\includegraphics[width=.8\textwidth]{img/nachhaltig-freq-1986-2016}
	\caption[corpus-size]{DWDS-Wortverlaufskurve für \lstinline|$l=/nachhaltig/i| im Zeitraum 1986–2016\protect\footnotemark}
	\label{fig:nachhaltig-freq-1986-2016}
\end{figure}

Die Verwendungshäufigkeit von \textit{nachhaltig} steigt von rund 17 Instanzen pro Million Tokens 1986 auf bis zu 41 IpM im Zeitraum 2006–2015 deutlich an. Damit tritt die Benennung in diesem Zeitraum etwa 2,4-mal häufiger auf als noch 20 Jahre zuvor. Bei einer Betrachtung einzelner Jahre (statt Zehnjahreszeiträumen wie in Tabelle und Diagramm) sticht das Jahr 2007 mit einem Wert von 74,91 IpM heraus. Im Jahr 2016 sinkt die Gebrauchshäufigkeit auf etwa 33 IpM ab.

\subsubsection{Kontextanalyse}

% Abbildung 6
	\footnotetext{Erstellt durch das Digitale Wörterbuch der deutschen Sprache: \urlwortverlaufD, abgerufen am 05.09.2017.}

Tabelle~\ref{tab:kollokationen-epoche4} zeigt die am stärksten mit \textit{nachhaltig} assoziierten Kollokationen im Zeitraum 1986–2016.

\begin{table}[h!]
	\centering
	\renewcommand{\arraystretch}{1.5}
	
	\caption{Top-Kollokationen von \lstinline|$l=/nachhaltig/i| in der Epoche 1986–2016\protect\footnotemark}
	\label{tab:kollokationen-epoche4}
	
	\begin{threeparttable}
	
	\begin{tabular}{cccc}
	\textbf{Rang} & \textbf{Lemma} & \boldmath{$f_{12}$} & \textbf{logDice} \\ \hline
	1 & Entwicklung & 1.428 & 5,62 \\ \hline
	2 & Wachstum & 506 & 5,28 \\ \hline
	3 & stören & 94 & 3,97 \\ \hline
	4 & ökologisch & 126 & 3,93 \\ \hline
	5 & Erholung & 100 & 3,92 \\ \hline
	\end{tabular} 
	
	\begin{tablenotes}
	\footnotesize
	\setlength{\itemindent}{-1.2em}
	\item $f_{12}$: absolute Kookkurenzfrequenz
	\item logDice: normalisiertes Assoziationsmaß
	\end{tablenotes}
	
	\end{threeparttable}
\end{table}

% Tabelle 7
	\footnotetext{Erstellt mit DiaCollo: \urlkollokD, abgerufen am 12.09.2017.}

War die Phrase „nachhaltige Entwicklung“ in der vorangehenden Epoche noch überhaupt nicht nachzuweisen, ist sie nun die stärkste Kollokation für \textit{nachhaltig}. Nach ersten Belegen ab 1991 tritt sie hochfrequent auf – teilweise in fachsprachlicher Bedeutung, wie die Korpusbelege~\ref{ex:1986–2016-entw-fach-1}–\ref{ex:1986–2016-entw-fach-n} zeigen.

\begin{exe}
\ex\label{ex:1986–2016-entw-fach-1} „Trotz solchen Bemühens um eine nachhaltige, ressourcenschonende Entwicklung findet im Süden eine Umweltzerstörung in dramatischem Umfang statt.“ (Die Zeit, 08.02.1991, Nr. 07)
\ex „Somit fehlt weiterhin ein Modell nachhaltiger, also dauerhafter und umweltverträglicher Entwicklung“ (Die Zeit, 20.03.1992, Nr. 13)
\ex „Ziel ist nicht etwa kurzfristiges Wachstum um jeden Preis, sondern nachhaltige Entwicklung, die nicht ihre eigenen Grundlagen auf Dauer zerstört.“ (Berliner Zeitung, 11.08.1994)
\ex „Wie kann es sein, dass die Generation, die die nachhaltige Entwicklung wirklich betrifft, bei deren Umsetzung nicht mitreden darf?“ (Zeit Campus, 15.04.2009, Nr. 03)
\ex\label{ex:1986–2016-entw-fach-n} „Es werde für mehr Wachstum und Arbeitsplätze sorgen, nachhaltige Entwicklung fördern sowie Produktivität und Lebensstandard sichern.“ (Die Zeit, 05.10.2015 (online))
\end{exe}

Allerdings wird die Benennung selbst in dieser Kollokation teilweise allgemeinsprachlich (im Sinne von \textit{dauerhaft}, \textit{längere Zeit anhaltend}) verwendet. Die Beispiele \ref{ex:1986–2016-entw-allg-1}–\ref{ex:1986–2016-entw-allg-n} belegen dies.

\begin{exe}
\ex\label{ex:1986–2016-entw-allg-1} „Die Entwicklungen auf dem Arbeitsmarkt werden nachhaltiger durch Tarifparteien, weltwirtschaftliche Einflüsse, Bundesregierung und Bundesbank oder Bundesgesetzgeber beeinflußt, als eine Landesregierung sie bewirken könnte.“ (Die Zeit, 27.06.1986, Nr. 27)
\ex „Die wirtschaftliche Entwicklung ist erfreulicherweise viel kräftiger und viel nachhaltiger, als es im Frühjahr den Anschein hatte.“ (Die Zeit, 07.10.2010 (online))
\ex\label{ex:1986–2016-entw-allg-n} „In einem Positionspapier warnen der Deutsche Skiverband (DSV) und der Snowboard-Verband Deutschland (SVD) das IOC davor, durch seine Entscheidungen ‚erfolgreiche Prozesse zur nachhaltigen und systematischen Entwicklung unterschiedlicher Sportarten‘ zu behindern.“ (Die Zeit, 06.11.2015 (online))
\end{exe}

Ähnlich verhält es sich bei der Kollokation „nachhaltiges Wachstum“. \ref{ex:1986–2016-wachs-fach-1} und \ref{ex:1986–2016-wachs-fach-n} geben Beispiele für eine fachsprachliche Verwendung im Sinne der Brundtland-Definition.

\begin{exe}
\ex\label{ex:1986–2016-wachs-fach-1} „Am Zuckerhut hatte auch Weltbank-Präsident Lewis Preston versprochen, seine Behörde müsse ‚eine führende Rolle bei der Einleitung einer neuen Ära internationaler Zusammenarbeit mit dem Ziel eines nachhaltigen, ökologisch verträglichen Wachstums einnehmen‘.“ (Die Zeit, 16.10.1992, Nr. 43)
\ex\label{ex:1986–2016-wachs-fach-n} „Wem es ernsthaft um ‚nachhaltiges Wachstum‘ geht, der muß für die Verminderung aller fossilen Energieträger eintreten – und dazu zählt zweifellos auch das Erdgas.“ (Die Zeit, 18.11.1994, Nr. 47)
\end{exe}

Ebenso wie „nachhaltige Entwicklung“ wird jedoch auch „nachhaltiges Wachstum“ häufig in einer eher allgemeinsprachlichen Bedeutung verwendet, insbesondere in Bezug auf Wirtschaftswachstum. Ein solcher Gebrauch lässt sich besonders frequent seit der Finanz- und Währungskrise ab 2007 beobachten. Korpusbelege dafür finden sich in \ref{ex:1986–2016-wachs-allg-1}–\ref{ex:1986–2016-wachs-allg-n}.

\begin{exe}
\ex\label{ex:1986–2016-wachs-allg-1} „Damit würden, so die Autoren der Studie, bessere Voraussetzungen für ein nachhaltiges Wachstum von Investitionen, Produktion und Beschäftigung bestehen als zu Anfang der sechziger Jahre.“ (Die Zeit, 31.12.1993, Nr. 01)
\ex „Eine stabile Währung sei eine unverzichtbare Grundlage für nachhaltiges Wachstum und mehr Beschäftigung.“ (Berliner Zeitung, 30.12.1996)
\ex „Langfristig strebe Pixelpark ein nachhaltiges und vor allem profitables Wachstum an.“ (Berliner Zeitung, 28.06.2001)
\ex\label{ex:1986–2016-wachs-allg-n} „Um wieder ein nachhaltiges Wachstum zu erlangen, seien weitere Schuldenerleichterungen notwendig.“ (Die Zeit, 25.09.2016 (online))
\end{exe}

Eindeutiger ist die Lage bei den übrigen Top-Kollokationen. In Zusammenhang mit dem Kollokat „stören“ wird \textit{nachhaltig} ausschließlich in seiner allgemeinsprachlichen Bedeutung gebraucht, wie \ref{ex:1986–2016-stören-1}–\ref{ex:1986–2016-stören-n} illustrieren.

\begin{exe}
\ex\label{ex:1986–2016-stören-1} „dann ist das Vertrauen zu den Eltern nachhaltig gestört“ (Die Zeit, 01.08.1986, Nr. 32)
\ex „Auch weiße Männerbeine mit Sandalen stören die Exotik nachhaltig“ (Die Zeit, 28.08.1987, Nr. 36)
\ex\label{ex:1986–2016-stören-n} „Die Verabschiedung des Papiers hat das Verhältnis zwischen Deutschland und der Türkei nachhaltig gestört.“ (Die Zeit, 28.07.2016 (online))
\end{exe}

In der Nachbarschaft von „ökologisch“ wird \textit{nachhaltig} dagegen fachsprachlich verwendet, wie \ref{ex:1986–2016-ökologisch-1}–\ref{ex:1986–2016-ökologisch-n} zeigen.

\begin{exe}
\ex\label{ex:1986–2016-ökologisch-1} „‚Wir haben das klare Ziel, ein ökologisch nachhaltig wachsendes Unternehmen zu werden‘, erklärt der für Umweltschutz zuständige Vizepräsident Robert Bringer.“ (Die Zeit, 15.05.1992, Nr. 21)
\ex „Das einfache Leben ohne Energie, Chemie, Mobilität, gestützt auf ein Minimum an Ressourcenverbrauch das ist nicht nur Steinzeitkommunismus, sondern auch ökologische Nachhaltigkeit auf niedrigstem Niveau.“ (Berliner Zeitung, 07.05.1998)
\ex „Die Informationsgesellschaft [...] könne nur überleben, wenn sie sich auf Dauer zur ökologischen Nachhaltigkeit bekenne.“ (Der Tagesspiegel, 07.07.1998)
\ex „Geprägt ist sie durch das Bestreben, wirtschaftliches Wachstum und Wettbewerb sozialverträglich und ökologisch nachhaltig zu gestalten.“ (Die Zeit, 13.12.2012, Nr. 51)
\ex\label{ex:1986–2016-ökologisch-n} „Vereint könnten Bayer und Monsanto noch mehr dazu beitragen, die stark wachsende Weltbevölkerung auf eine ökologisch nachhaltige Weise zu ernähren.“ (Die Zeit, 15.09.2016 (online))
\end{exe}

Gemeinsam mit dem Kollokat „Erholung“ wird \textit{nachhaltig} schließlich wieder allgemeinsprachlich gebraucht, besonders in Bezug auf Wirtschaft, Wertpapiere, Rohstoffpreise und Währungskurse. Die Korpusbelege \ref{ex:1986–2016-erholung-1}–\ref{ex:1986–2016-erholung-n} geben einige typische Verwendungsbeispiele.

\begin{exe}
\ex\label{ex:1986–2016-erholung-1} „Auf eine nachhaltige Erholung der Kakaopreise ist kaum noch zu hoffen“ (Die Zeit, 25.05.1990, Nr. 22)
\ex „Im Grunde traut er – trotz markanter Aufwärtsbewegung – den Börsen eine nachhaltige Erholung gar nicht zu.“ (Die Zeit, 15.03.1991, Nr. 12)
\ex „Sie unterstütze weiter eine nachhaltige Erholung des Dollar zum Yen.“ (Berliner Zeitung, 23.09.1995)
\ex „Mögliche Hoffnungen auf eine nachhaltige Erholung wurden allerdings von der New Yorker Wall Street enttäuscht.“ (Berliner Zeitung, 19.04.1997)
\ex „Eine schnelle nachhaltige Erholung des Euro wird derzeit nicht erwartet.“ (Berliner Zeitung, 12.08.2000)
\ex „Die wirtschaftliche Erholung ist aus Sicht der EZB noch nicht nachhaltig.“ (Die Zeit, 08.10.2009, Nr. 42)
\ex\label{ex:1986–2016-erholung-n} „Viele Ökonomen meinen, dass der US-Arbeitsmarkt für eine nachhaltige wirtschaftliche Erholung um monatlich rund 200.000 Stellen wachsen müsse.“ (Die Zeit, 07.02.2014 (online))
\end{exe}

Die Kontextanalyse zeigt, dass \textit{nachhaltig} in jüngster Zeit sowohl fachsprachlich (im Sinne der Brundtland-Definition) als auch allgemeinsprachlich (in der Bedeutung \textit{dauerhaft}, \textit{längere Zeit anhaltend}) verwendet wird und von der Allgemeinsprache aus offenbar auch Eingang in andere Fachsprachen (insbesondere die der Ökonomie) findet. Dabei ist die jeweilige Bedeutungsvariante in manchen Kontexten einheitlich (wie bei „stören“, „ökologisch“ oder „Erholung“), in anderen jedoch gemischt (siehe „Entwicklung“ und „Wachstum“). Dies kann zu einer Ambiguität führen, die in bestimmten Kontexten kaum aufzulösen ist. \ref{ex:1986–2016-ambig-1} und \ref{ex:1986–2016-ambig-n} zeigen derlei ambige Beispiele.

\begin{exe}
\ex\label{ex:1986–2016-ambig-1} „Die Demokratische Linke will eine lange Amtszeit für eine nachhaltige Politik und pocht auf die Rechte von Migranten.“ (Die Zeit, 19.06.2012, Nr. 25)
\ex\label{ex:1986–2016-ambig-n} „Die CeBIT hat sich in diesem Jahr das Thema ‚Datability‘ auf die Fahnen geschrieben - ein Kunstwort aus dem Begriff Big Data und englischen Worten wie ability, sustainability, responsibility, also den Möglichkeiten der nachhaltigen und verantwortungsvollen Nutzung großer Datenmengen.“ (Die Zeit, 16.01.2014 (online))
\end{exe}

Diese Mehrdeutigkeit wird in Einzelfällen sogar explizit angesprochen, wie in Korpusbeleg~\ref{ex:1986–2016-ambig}.

\begin{exe}
\ex\label{ex:1986–2016-ambig} „Und Fondsmanager haben es immer öfter mit Anlegern zu tun, die Wert auf nachhaltige Geldanlagen legen – nicht nur was den Gewinn betrifft.“ (Der Tagesspiegel, 03.06.2004)
\end{exe}

\section{Diskussion}
\label{sec:diskussion}

In diesem Abschnitt sollen die Ergebnisse der Analysen auf die Fragestellungen aus Abschnitt~\ref{sec:ziele} zurückgeführt und eingeordnet werden.

\begin{enumerate}[rightmargin=1cm]
\item[\textbf{F1}] Wie hat sich die Verwendungshäufigkeit der Benennung \textit{nachhaltig} seit ihrem ersten Auftreten entwickelt?
\end{enumerate}

Die Frequenzauswertung in Abschnitt~\ref{subsec:freq-gesamt} belegt – wenig überraschend – einen deutlichen Anstieg der Verwendungshäufigkeit von \textit{nachhaltig} und verwandten Benennungen seit Erscheinen des Brundtland-Berichts, mit einem Maximum im Zeitraum 2006–2015. Zuvor war die Benennung seit ihrem ersten Korpusbeleg 1779, also über einen Zeitraum von rund 200 Jahren, mit relativ konstanter Frequenz verwendet worden. Dies allein ist allerding noch kein Beleg für einen inflationären, also mit einem Wert- oder Bedeutungsverlust einhergehenden, Gebrauch des Nachhaltigkeitsbegriffs.

Ob der Abfall der normalisierten Verwendungsfrequenz im Jahr 2016 Ausdruck einer üblichen Frequenzschwankung ist oder aber eine Trendwende anzeigt und signalisiert, dass die Verwendung von \textit{nachhaltig} wieder aus der Mode kommt, kann zum jetzigen Zeitpunkt allerdings noch nicht beurteilt werden.

\begin{enumerate}[rightmargin=1cm]
\item[\textbf{F2}] Wie haben sich in diesem Zeitraum die Kontexte verändert, in denen \textit{nachhaltig} auftritt?
\end{enumerate}

\textit{Nachhaltig} wird in allen untersuchten Epochen in seiner ursprünglichen Bedeutung im Kontext der Forstwirtschaft verwendet. Gleichzeitig wird die Benennung schon früh auf andere natürliche Ressourcen (z.\,B. Tierpopulationen in Bezug auf die Jagd oder Fischerei) übertragen – ein erster Beleg hierfür datiert auf 1835.

Daneben und – zumindest im untersuchten Korpus – hauptsächlich wird \textit{nachhaltig} allerdings in seiner allgemeinsprachlichen Bedeutung in den verschiedensten Kontexten gebraucht. Einen Schwerpunkt bildet hier sicherlich die Beschreibung wirtschaftlicher bzw. finanzieller Methoden und Entwicklungen.

Im Zeitraum 1846–1915 wird \textit{nachhaltig} außerdem auch in den Bedeutungen \textit{permanent}, \textit{fest} (in Bezug auf Arbeit oder Einkommen) und \textit{ausdauernd} verwendet – ein Gebrauch, der in dieser Form später nicht mehr nachgewiesen werden kann.

Die durch die Vereinten Nationen forcierte Einführung des Konzepts der \textit{nachhaltigen Entwicklung} in den internationalen Diskurs schlägt sich auch auf das untersuchte Korpus nieder. Ab den frühen 1990er Jahren werden die Kollokationen „nachhaltige Entwicklung“, „nachhaltiges Wachstum“ und „ökologische Nachhaltigkeit“ zu häufig verwendeten Schlagworten.

\begin{enumerate}[rightmargin=1cm]
\item[\textbf{F3}] Wird die Benennung \textit{nachhaltig} in verschiedenen Zeitabschnitten eher in ihrer fachsprachlichen Bedeutung (im Sinne der Brundtland-Definition), in ihrer allgemeinsprachlichen Bedeutung (als Synonym zu \textit{dauerhaft}, \textit{längere Zeit anhaltend}) oder gemischt verwendet?
\end{enumerate}

Ein überraschendes Ergebnis dieser Studie ist sicherlich, wie früh die Benennung \textit{nachhaltig} bereits in einer allgemeinsprachlichen Bedeutung verwendet wird. Erste Belege hierfür finden sich ab 1779. Wenn ein Gebrauch im fachsprachlichen forstwissenschaftlichen Kontext auch durchgängig nachzuweisen ist, so überwiegt die allgemeinsprachliche Verwendung quantitativ doch im gesamten Zeitraum der Untersuchung deutlich.

Eine weitere interessante Beobachtung ist, wie \textit{nachhaltig} selbst in Fachpublikationen sowohl fach- als auch allgemeinsprachlich verwendet wird. Dies ist bereits im 19. Jahrhundert zu beobachten und führt in einigen Beispielen zu einer nicht einfach auflösbaren Ambiguität.

Diese Ambiguität wird in der jüngsten untersuchten Epoche von 1986 bis 2016 besonders deutlich, wo \textit{nachhaltig} sowohl in seiner allgemeinsprachlichen Bedeutung im wirtschaftlichen Kontext als auch in seiner fachsprachlichen Bedeutung im Sinne der Brundtland-Definition häufig verwendet wird. Bei Kollokationen wie „nachhaltige Entwicklung“ und „nachhaltiges Wachstum“ muss je nach Kontext unterschieden werden, welche Definition dem verwendeten Nachhaltigkeitsbegriff zugrunde liegt.

Mit Blick auf die in Abschnitt~\ref{subsec:kritik} zusammengefasste Kritik an der Verwendung des Nachhaltigkeitsbegriffs lässt sich feststellen, dass eine allgemeinsprachliche Verwendung der Benennung kein neues Phänomen ist, das erst mit ihrer wachsenden Gebrauchshäufigkeit aufgetreten ist, sondern bereits seit den ersten Korpusbelegen aus dem späten 18.~Jahrhundert durchgängig zu beobachten ist. \textit{Nachhaltig} ist, in der Bedeutung von \textit{dauerhaft} oder \textit{längere Zeit anhaltend}, also seit Langem fester Bestandteil der Allgemeinsprache.

Darüber hinaus ist es keine Seltenheit, dass Fachbegriffe Eingang in die Allgemeinsprache finden und dort in einer weniger spezifischen Bedeutung verwendet werden. Viel problematischer scheint es mir, dass \textit{nachhaltig} selbst in den beteiligten Fachkreisen oft weder besonders spezifisch noch mit einer klaren und einheitlichen Definition verwendet wird. Wenn die Benennung in ein und derselben Publikation sowohl in ihrer fach- wie in ihrer allgemeinsprachlichen Lesart gebraucht wird, führt dies zu zusätzlicher Verwirrung und kann letztlich zur beklagten Verwässerung des Begriffs beitragen.

Selbst wenn eine inflationäre Verwendung oder gar bewusste Verwässerung des Nachhaltigkeitsbegriffs durch diese Studie also nicht nachgewiesen werden kann, so stellt der seit langer Zeit etablierte breite Gebrauch der Benennung \textit{nachhaltig} doch ein Hindernis dar, weiten Teilen von Politik, Wirtschaft und Gesellschaft die wirkliche Bedeutung des Nachhaltigkeitskonzepts mit ihren weitreichenden Auswirkungen zu vermitteln. Dies mag zum Bestreben geführt haben, \textit{nachhaltig} durch einen weniger vorbelasteten Neologismus zu ersetzen: Im politischen Diskurs wird in jüngster Zeit \textit{enkelgerecht} vermehrt als Synonym verwendet.

\section{Zusammenfassung}
\label{sec:zusammenfassung}

Angestoßen durch die verbreitete Kritik an einer inflationären Verwendung  des Nachhaltigkeitsbegriffs wurde in dieser Arbeit die Entwicklung von \textit{nachhaltig} und verwandten Benennungen im Zeitraum von 1776 bis 2016 mit korpuslinguistischen Mitteln untersucht. Dabei wurden besonders die Verwendungshäufigkeiten, Kontexte und Kollokationen der Benennung diachron analysiert.

Die Studie zeigt, dass \textit{nachhaltig} seit den 1990er Jahren tatsächlich wesentlich häufiger verwendet wird als in den 200 Jahren zuvor. Dass mit diesem Frequenzanstieg auch eine Verwässerung des Begriffs im Sinne einer diffusen allgemeinsprachlichen Bedeutung einhergeht, kann allerdings nicht bestätigt werden. Vielmehr zeigt die Analyse, dass die Benennung bereits seit dem späten 18. Jahrhundert allgemeinsprachlich als Synonym zu \textit{dauerhaft}, \textit{längere Zeit anhaltend} gebraucht wird und in dieser Bedeutung etabliert ist.

Nicht zuletzt trägt die Fachliteratur durch das Fehlen einer klaren und einheitlichen Definition und eine wenig spezifische Verwendung der Benennung, teilweise auch in ihrer allgemeinsprachlichen Lesart, zu einem wahllosen Gebrauch des Nachhaltigkeitsbegriffs bei, was angesichts der zentralen Bedeutung des Nachhaltigkeitskonzepts als alle Lebensbereiche umfassendes Entwicklungsziel ein durchaus kritikwürdiger Zustand ist.

\newpage
\begin{thebibliography}{9}

\bibitem[Dudenredaktion(2017)]{Duden2017} Dudenredaktion (2017): \textit{„Nach­hal­tig­keit“ auf Duden online.} \url{http://www.duden.de/node/658572/revisions/1337271/view}. Abgerufen am 31.08.2017.
\bibitem[Hartig(1804)]{Hartig1804} Hartig, G. L. (1804). \textit{Anweisung zur Taxation der Forste oder zur Bestimmung des Holzertrags der Wälder. Erster oder theoretischer Theil.} (2., ganz umgearbeitete und vermehrte Aufl.). Gießen und Darmstadt: Georg Friedrich Heyer.
\bibitem[Jurish(2015)]{Jurish2015} Jurish, B. (2015). DiaCollo: On the trail of diachronic collocations. In \textit{Proceedings of the CLARIN Annual Conference}, 28–31.
\bibitem[Geyken(2007)]{Geyken2007} Geyken, A. (2007). The DWDS corpus: A reference corpus for the German language of the 20th century. In: Fellbaum, C. (Hg.): \textit{Collocations and Idioms: Linguistic, lexicographic, and computational aspects.} London, 23–41.
\bibitem[Geyken et al.(2011)]{Geyken2011} Geyken, A., Haaf, S., Jurish, B., Schulz, M., Steinmann, J., Thomas, C., \& Wiegand, F. (2011). Das Deutsche Textarchiv: Vom historischen Korpus zum aktiven Archiv. In \textit{Digitale Wissenschaft}, 157–162.
\bibitem[Görg(1996)]{Görg1996} Görg, C. (1996). Sustainable Development–Blaupause für einen ‘ökologischen Kapitalismus’. In: Brentel, Helmut u.a. (Hg.): \textit{Gegensätze. Elemente kritischer Theorie.} Frankfurt/New York, 178–193.
\bibitem[Kastenholz et al.(1996)]{Kastenholz-et-al1996} Kastenholz, H. G., Erdmann, K. H., \& Wolff, M. (1996). Perspektiven einer nachhaltigen Entwicklung – eine Einführung. In \textit{Nachhaltige Entwicklung}. Berlin/Heidelberg: Springer, 1–8.
\bibitem[Ninck(1997)]{Ninck1997} Ninck, M. (1997). \textit{Zauberwort Nachhaltigkeit.} Zürich: vdf, Hochschulverl. AG an der ETH.
\bibitem[Ohlmeier~\& Brunold(2015)]{OhlmeierBrunold2015} Ohlmeier, B., \& Brunold, A. (2015). \textit{Politische Bildung für nachhaltige Entwicklung: Eine Evaluationsstudie.} Springer.
\bibitem[United Nations(1987)]{UN1987} United Nations. (1987). \textit{Our Common Future: Report of the World Commission on Environment and Development (A/42/427).} \url{http://www.un-documents.net/ocf-02.htm}. Abgerufen am 14.09.2017.
\bibitem[Vogel(2011)]{Vogel2011} Vogel, T. (2011). \textit{Naturgemäße Berufsbildung – Gesellschaftliche Naturkrise und berufliche Bildung im Kontext Kritischer Theorie.} Norderstedt bei Hamburg: Books on Demand.
\bibitem[Zürcher(1965)]{Zürcher1965} Zürcher, U. (1965). \textit{Die Idee der Nachhaltigkeit unter spezieller Berücksichtigung der Gesichtspunkte der Forsteinrichtung.} Dissertation, ETH Zürich.

\end{thebibliography}

\end{document}
